
\section{OGIE}
\label{ogie}

\subsection{Get started}
To start \emph{OGIE} select \menu{Tools>OGIE -- 3d} in \emph{GPLIGC},
or type \texttt{ogie igcfile.igc}
at the command line.% (xterm, console):\\% or MS-DOS prompt): \\

There are three different modes to use \emph{OGIE}:

\begin{itemize}

\item IGC-file mode:
You can give an igc-file as a single argument.
If you use more than one argument on the commandline, you need to specify
the igc-file by adding  \texttt{--igc-file FILENAME} (or \texttt{-i FILENAME}).

\item Terrain viewer: Select the centre of your view with  \texttt{--lat} and \texttt{--lon}. The size of the area can be selected with the following options: \texttt{--border, --border-lat, --border-lon}

\item GPS live view: Use the option \texttt{--gpsd}. \emph{ogie} will connect to the local gpsd \cite{gpsd} and obtain positional information for a live display of your location.

\end{itemize}

\subsection{Menus}
The pop-up menu is accessible by pressing the right mouse button in the \emph{OGIE} window. Most important options can be changed here.


\subsection{Mouse control}
The direction of view can be controlled with the mouse, the mouse pointer is invisible and cannot leave the window, unless
mouse control is disabled by pressing m.

Moving the mouse while the left button is pressed will result in rotating your position around the centre of the scene or around
the position of the marker (if activated). Dragging the mouse up and down, with the middle button pressed, will shift your
position towards or away from the centre or the marker position (if marker is activated).


Moving while the middle and left mouse button is pressed will shift the scene.


\subsection{Joystick control}
%A joystick is supported on Windows platforms via GLUT.
On Unix/Linux (X11) the joystick can not accessed via GLUT (because
GLUT never supported joysticks on X11). If you want to use your joystick on X11, you have to install freeglut \cite{freeglut}.

The joysticks x,y and z-axis will move the viewpoint to the side, forward-backward and up-down. How much the viewpoint
will be shifted can be set in the configfile (JOYSTICK\_FACTOR\_X,Y,Z, see \ref{config}).

\subsection{Keyboard control}
For information on the Keyboard functions you should read the section \ref{keys}.
In the pop-up menu a \emph{help} is present, which will show the most important keys.
If you like to change the controls, edit in \texttt{KeyPressed} and \texttt{specialKeyPressed}-functions in
\texttt{GLexplorer.cpp} and recompile.

If you need the mouse pointer, it can be made visible by pressing key \emph{m}.


\subsection{GPS live mode}
\label{gps}

The commandline arguments \texttt{--gpsd}, \texttt{--gpsd--server=STRING} or \texttt{--gpsd-port=INT} enable the GPS live mode.
\emph{ogie} will connect to a gpsd at server:port and retrieve the position.
The default server is localhost and the default port is 2947.
Subsequently, a track is build up by the GPS information.
If the Movie-Mode (see section~\ref{movie}) is enabled, the marker is always kept at the actual position.
Otherwise the marker can be moved freely, as with an IGC-file.
The info display (see section~\ref{info}) shows some additional information:
Sat/Mode: number of used satellites, GPS mode (2D or 3D). eph/epv: estimated horizontal and vertical errors. Interruptions: count of interruption of the GPS signal.

To use this mode \emph{ogie} has to be build with gpsd support.

\paragraph{Example}
I have my Germin Geko301 (serial) with an serial-to-usb adapter conntected to my laptop. The Geko is set to NMEA-mode. From my \texttt{messages} I know, that the serial-to-usb adapter is at \texttt{/dev/ttyUSB2}. Gpsd is easy to invoke \texttt{sudo /usr/sbin/gpsd /dev/ttyUSB2}.
Now, \texttt{ogie --gpsd} starts the fun!

\subsection{Waypoints}
\label{wp}

Waypoints can be displayed by ogie.
The file containing the waypoints can be declared in the ogie-configuration file by the keyword \texttt{WAYPOINTS\_FILE} or by a commandline argument \texttt{--waypoints-file}. Some more keywords and command-line arguments are available to change the default behaviour.

To switch them on or off use the F12 key. Page-up and page-down can be used to change the size of the spheres, the text-size can be changed with shift-page-up/down. Using shift-pos1 or shift-end changes the displayed text (waypoint-long name, waypoint short-name, waypoint-altitude, waypoint-symbol name).

\subsubsection{Format of the waypoint file}
As there are probably hundreds of waypoint file formats available, I chose a simple one, which I use with my handheld Garmin GPS and \emph{gpsbabel} \cite{gpsbabel}.
It has six columns of data: latitude (degrees), longitude (degrees), altitude (metres), short name (max six letters), long name, symbol name.
Columns are seperated by whitespaces (therefore no whitespaces are allowed within the names).
Using \emph{gpsbabel} it should be easy to convert any format to this.
You'll find a gpsbabel-style (\texttt{gpligcwpt.gpsbabelstyle}) file in the PREFIX/share/gpligc folder.
Here is an example how to convert a cambridge waypoint file to the needed format:\\

\texttt{gpsbabel -i cambridge -f cambridgefile  -o xcsv,style=gpligcwpt.gpsbabelstyle -F mywpts.gwpt}\\
The important part is the output format option `xcsv,style=' using the provided style file.\\
However, \emph{gpsbabel} is cool, you should have a look at it anyway.
You can even download your waypoints from a Garmin device like this:\\
\texttt{gpsbabel -i garmin -f /dev/ttyS1 -o xcvs,style=gpligcwpt.gpsbabelstyle -F outfile.gwpt}


%$Id: airspace.tex 3 2014-07-31 09:59:20Z kruegerh $
%spellch 1.8
%;;; Local IspellDict: "british"


\subsection{Airspace}
\label{airspace}

If you want airspace information to be displayed, you should get an OpenAir\texttrademark\ airspace file (that's the same format as used by Winpilot) for your region and set up your \texttt{.ogierc} file.
One keyword declares  the filename of the airspace-file, another one sets the default, whether airspaces should be displayed or not. \\
\texttt{OPEN\_AIR\_FILE /path/to/OpenAir/file} \\
\texttt{AIRSPACE true} \\
An alternative way are the following command-line options \\
\texttt{--airspace-file=/path/to/OpenAir/file} and \texttt{--airspace} or \texttt{--no-airspace} to turn them on or off.

At runtime, airspaces can be switched on or off via the menu or by F9. Shift-F9 toggles the wire frame and transparent mode.

\subsubsection{How and where to get OpenAir files}
On the gpligc web-site you may find an airspace folder in the download area. Some OpenAir formatted files can be found there.
Another option is the page of J. Leibacher \cite{leibacher}.

% outdated
%\emph{I don't know whether the following paragraph is still valid!}\\
%Another option is to use the program of \texttt{Carl Ekdahl}, which can be found on the \emph{Soaring Server}.
%That program works on M\$-Windows only, but can create OpenAir formatted files from recent \emph{DAFIF} sources.
%To do that the dafift.zip is needed and the following files need to be copied to the right location inside the airspace-program %(from \texttt{Carl Ekdahl}) folders: \texttt{BDRY.TXT, BDRY\_PAR.TXT, SUAS.TXT, SUAS\_PAR.TXT}.




%%% Local Variables:
%%% mode: latex
%%% TeX-master: "GPLIGC_manual.tex"
%%% End:



%$Id: dem.tex 3 2014-07-31 09:59:20Z kruegerh $
%;;; Local IspellDict: "british"


\subsection{Digital Elevation Model}
\label{dem}

There are many digital elevation models on the web, which can be downloaded for free and used with ogie:
ETOPO2 \cite{etopo2}, GLOBE \cite{globe}, GTOPO30 \cite{gtopo30}, SRTM30 Plus (TOPO30) \cite{srtm30plus}, SRTM30 \cite{srtmv2}, SRTM-3 \cite{srtmv2} and SRTM-1 \cite{srtmv2}.
GTOPO30, SRTM30 (Plus) and GLOBE have a resolution of 30 arc-seconds, 1km.
ETOPO2 has a 2-minute grid (4km), but also covers the oceans.
SRTM-3 is 3 arc-seconds (90m) and SRTM-1 (only available for U.S.) is 1 arc-second (30m).

The "Shuttle Radar Topography Mission" (SRTM) topographic data
with resolution 1 arc-second (for USA) and 3 arc-second for almost the rest of the world
is available, but due to the high resolution not very good for regular flight-analysis. (Graphics
hardware won't handle larger areas).

My recommendation is to use either GTOPO30 or SRTM30 (Plus). 
If you have lots of hard disk space and want to analyse flights from many countries, you should consider
to build a \texttt{WORLD.DEM} from GTOPO30/SRTM30. If you like to explore the oceans, you may merge
it with ETOPO2 data. Data of this type is available from the gpligc download directories for many countries (including needed configuration settings).

SRTM-3 and SRTM-1 can be used for small-scale high-resolution application.


\subsubsection*{Data format}

Binary data in 2 byte integer (big endian byte) format is needed.
(You can get these directly from GLOBE, GTOPO30 and ETOPO2 Web sites)
Little endian data can be used with the config option: \texttt{BIGENDIAN false}.


\subsubsection{GTOPO30, SRTM30}
The worldwide GTOPO30 \cite{gtopo30} elevation model is split up in 33 pieces (tiles).
Get the "tile" you need and put the full path to the *.DEM file into the
configuration-file (see \ref{demconf}). You also need to set the rows and columns
and minima and maxima and grid resolution.

If you have lots of space on your hard-disk and a fast internet connection you should consider to
get all (33) tiles (about 280~MB compressed) and use the \texttt{createworld}
tool to generate a \texttt{WORLD.DEM} (single file containing worldwide elevation data, really cool!) file:

\begin{enumerate}
\item "\texttt{tar xvzf}"  all tiles into one directory (taht will need more than 2 GB).
      You only need to extract the *.DEM files from the *.tar.gz archives
      downloaded from GTOPO. Use the following cmdline (in the directory with
      all archives) to extract *.DEM files only: \\
      \texttt{find . -name '*0.tar.gz' -exec tar xvzf \{\}  *.DEM ';'}

\item invoke \texttt{createworld} in the same directory (this will need another 1.8 GB)
      
\item enjoy the 1.8 GB (!) \texttt{WORLD.DEM} (check \texttt{WORLD.DEM} for its size:
      should be 1.866.240.000 bytes)

\item settings for the \texttt{WORLD.DEM} can be found in the default-config file

\end{enumerate}

Now there is an improved SRTM30 model, which is based on the shuttle radar topography
mission. Basically, the SRTM30 seems to be a better GTOPO30. The SRTM30 data is also available
for free  and can be used to build the \texttt{WORLD.DEM} as described above. SRTM30 data didn't cover
the regions south of 60$^\circ$S. To obtain a \texttt{WORLD.DEM} file you should take the 6 arctic
tiles from GTOPO30, the remaining 27 from SRTM30.


\subsubsection{ETOPO2 (and merging it into the GTOPO30)}
If you have created a \texttt{WORLD.DEM} (1.8GB) datafile as described above, you can merge it with the bathymetry(sea-depth)-data from etopo20 \cite{etopo2}.
Get the \texttt{etopo20.i2.gz} file from the web. "Gunzipped" it has 116.672.402 bytes.
Put the \texttt{WORLD.DEM} and the \texttt{etopo2.i2} in the same directory and call (within that directory) \texttt{etopo2merger}, which will merge them into a \texttt{WORLD3.DEM} file.
Because the ETOPO2 resolution is lower than the resolution from GTOPO30, the additional data-points are obtained by interpolation.

\subsubsection{GLOBE}
Download the region you need (freely selectable \cite{globe}) and make sure that you
get the right data format. In the *.hdr file (which you will get too,
you can find all needed information to edit the config-file.

These are the options to be selected at GLOBE download page:\\
FreeForm ND\\
int16\\
Mac/Unix Binary\\

the data file is called   *.bin
the *.hdr file contains some information you need to edit the
ogie configfile.


\subsubsection{SRTM30 Plus (TOPO30)}
The SRTM30 Plus \cite{srtm30plus} elevation model is a merged SRTM30 and GTOPO30, including bathymetry data from several sources.
It can be downloaded as a single (1.8GB) data file from \cite{srtm30plus}.
Notice the different settings for DEM\_LAT\_MAX and DEM\_LON\_MIN! (differing from what should be used for SRTM30 and GTOPO30 world files).
See example configuration file!

\subsubsection{SRTM-1 and SRTM-3}
SRTM-3 (3 arc-seconds) data is available for free (for north and south-America and for Eurasia). SRTM-1
(1 arc-second) is available for the USA. The data is in .hgt format which is exactly, what ogie
can read. But the data is tiled into 1x1 degree pieces. This might be useful for high resolution analysis of
some terrain detail, but is just too much data for normal (glider-)flight analysis.
However, you'll find it at \cite{srtmv2}.
The Documentation folder will give you important information about data-format etc.


\subsubsection{SRTM-1 and SRTM-3 finished from seamless server}

From the usgs seamless server \cite{seamless} you can get these data.
It can be downloaded in a binary .bil format, which is accompanied by a .blw file, which contains additional information. Attention, there is a half-pixel shift.
The actual coordinates for the upp-left corner can be found in the last two lines in the .blw file.
It seems that void areas are set to 0, in contrast to the original \emph{research grade SRTM data} which uses -32768.
Additionally you will need \texttt{BIGENDIAN false}.


\subsubsection{USGS DEM (30-m and 10-m)}
\emph{This section is written by} \textsc{Vit Hradecky,} \emph{thanks} \\

Digital elevation data with 30-m and 10-m resolution for the U.S. is
now available for free at \cite{geocomm}.
The data is broken up into the standard USGS 7.5-min quads. Most of the
data is in the newer SDTS format, while some of it is in the older ASCII
DEM format. Fortunately, a utility exists to dump either into a raw
binary file, which is readable by ogie. Compile the C source from
\cite{readdem} and execute \\
\texttt{read\_dem berlin10m.DEM.SDTS.TAR berlin10m.BIN berlin10m.HDR 0} \\
This will convert data for the Berlin USGS quad from the SDTS format to
the 16-bit binary format, output to berlin10m.BIN, dump the headers into
berlin10m.HDR, and set bad data to zero elevation. The output will be in
little-endian byte order. Use the \texttt{BIGENDIAN false} option in the
configuration file to read the data correctly. The DEM latitude and
longitude limits can be found in berlin10m.HDR.
The 10-m and 30-m data is in units feet rather than meters. Use \texttt{DEM\_INPUT\_FACTOR
0.30488} in the configuration file.


\subsubsection{Configuring ogie for DEM}
\label{demconf}

For details on the configuration file see \ref{config}.
This section will only describe the settings for the digital elevation model setup.

In the configuration file you need to specify the following lines: \\
The full path to the used digital elevation data file: \\
\texttt{DEM\_FILE  /full/path/to/demfile/W020N90.DEM} \\

The number of rows and columns of data \\
\texttt{DEM\_ROWS 6000} \\
\texttt{DEM\_COLUMNS 4800} \\

The maxima and minima of your DEM-File \\
\texttt{DEM\_LAT\_MIN 40} \\
\texttt{DEM\_LAT\_MAX 90}\\
\texttt{DEM\_LON\_MIN -20}\\
\texttt{DEM\_LON\_MAX 20}\\

And the resolution (0.00833333 for GTOPO30 and GLOBE)\\
\texttt{DEM\_GRID\_LAT 0.008333333333}\\
\texttt{DEM\_GRID\_LON 0.008333333333}\\
Divide by 10 for SRTM-3.

For other config-file options see \ref{config}.


%%% Local Variables:
%%% mode: latex
%%% TeX-master: "GPLIGC_manual.tex"
%%% End:


\subsection{Terrain viewer mode}

\emph{OGIE} can be used without IGC-Files (as a Terrainviewer).
Give the centre of the area (which you want to watch) in decimal degrees \\
\texttt{--lat 53.5  --lon 8.5} \\
as commandline parameters to \texttt{ogie}, negative values for southern and western  hemisphere.
With the argument \texttt{--border km}, the half sidelength of terrain in kilometers can be set. You may specify the borders separately with \texttt{--border-lat km}  and \texttt{--border-lon km}.
If you want to watch very large areas, you can use
\texttt{--downscaling n}
where n is an integervalue bigger than 1. This will force the program to use
only every n-th datapoint from the elevation model.



\subsection{Colourscaling}
\label{color}

The following colourmaps are available:

\begin{enumerate}
\item  red - rainbow - white
\item  green - red - white
\item  black - white
\item  dark green - red
\item  magenta - light blue
\item  black - rainbow - white
\item  white
\item  black - red - yellow - white
\end{enumerate}

2 Colourmaps are used for terrain colourscaling. One upper (normal) colourmap and another (lower) colourmap for the terrain beneath sealevel. The value taken as sealevel can be set by the commandline switch \texttt{--sealevel m}. The colourmaps to be used can be set by \texttt{--colormap-sea n} and \texttt{--colormap n}. The upper colourmap can changed interactively by pressing keys 1-6, the lower (sea) colourmap can be changed with F10 and F11.
The default colourmaps can be set in the configuration file:
\texttt{COLORMAP n} and \texttt{COLORMAP\_SEA}
\\
Yes, the spelling of \emph{colour} in all parameters and cmdline options is \emph{color} [amer.]

\paragraph{Optimising colourmaps}
By default the colourmaps scale their colour-ranges from minimum to NN,
and from NN to maximum elevation (of displayed terrain).
If you want some more aggressive colourscaling you can
specify the minimum and maximum by giving \texttt{--colormap-min m} and \texttt{--colormap-max m}
Arguments are heights in Meters.
If the sealevel is outside the range (min, max) only one of the colourmaps will
be used.

\paragraph{Example}
\texttt{--colormap-sea 1 --colormap 3 --colormap-min 20 --colormap-max 3500 --sealevel 600}
This will cause \emph{OGIE} to use first colourmap between 20m and 600m, the gray-map between 600m and 3500m.

\paragraph{Sealevel2}
If you prefer a flat blue ocean surface instead of seafloor-terrain:
Setting a \texttt{--sealevel2 n}, will cause the explorer to set a ocean-like flat blue
surface at an elevation of n meters.

\paragraph{Sealevel3}
Almost like seavel2, but sealevel3 will be a transparent surface, through which the seafloor can be seen.
Setting a \texttt{--sealevel3 n}, will cause the explorer to set a ocean-like flat transparent blue
surface at an elevation of n meters.



\subsection{Maps}
If you have defined map-sets (as described in \ref{maps}) and turn them on,
(\texttt{--map} or \texttt{MAP true}, by menu (\menu{Maps on/off}) or pressing \keys{b}) then only the terrain covered by the
defined maps will be displayed. If you're using an elevation model as well, the
maps will be put on the surface (if the terrain-mode is active: \texttt{--landscape},
\texttt{LANDSCAPE true}, or activated by \menu{Terrain on/off} or pressing \keys{l}).
The modulation mode (or coloured map mode) can modulate the maps with colourscaling (on/off \keys{F8}, or \menu{Colored maps on/off} from menu. Another way would be to put \texttt{MODULATE on} in the configuration file).

%$Id: maps.tex 3 2014-07-31 09:59:20Z kruegerh $
%;;; Local IspellDict: "british"

\subsubsection{Setting up digitised maps}
\label{maps}

Since version 1.2 the digitised maps can be in jpeg format.
The file extension should be .jpg (not .JPG or .jpeg etc).
The old rgb-format texture maps can be used too, but jpg maps should be preferred (they do not need that much diskspace).
Since version 1.3 the \texttt{NUMBER\_OF\_MAPS} is not needed anymore.


\paragraph{How to prepare maps}
First of all you have to use a scanner or digital camera to get your maps into the computer (or just download stuff from the web).
To avoid differences due to projections, the map should not be in one big tile, but many small pieces. The smaller the better. For a 1:500.000 map (like ICAO) pieces of 40' x 40' are a good choice (1$^\circ$ x 1$^\circ$ is probably also OK).
For the further processing of the digitised maps a good image manipulation software is needed such as Gimp (the GNU image manipulation program \cite{gimp}).
The pieces have to be cut out from the scanned raw image(s).
Then the pieces have to be straighten out, to avoid any distortions. The latitude or longitude should be constant for each border.
I use the transform tool (which can be used to straighten out perspective
distortions etc) to define a (distorted) box along the gridlines of the
40'x40' box, as exact as possible. The transform tool will straighten this
out to a perfect rectangular box: the map-tile, which should be scaled to
some power-of-2 width and height (128x256 or 256x512 or 512x512 or 512x1024
or or...) otherwise this has to be done internally in ogie, which will slow down things a little.
Furthermore, you need to know the coordinates of each border.
Then save the map tile as jpg image.

\paragraph{Set up the \texttt{.ogierc} file}

For each map-tile the full path to the image-file and the coordinates of the top, bottom, left and right border have
to be given in the configuration file.
You need to have a section (as follows) for \emph{each} map-tile:

\begin{verbatim}
   MAP_FILE /usr/local/gpligc/maps/bremen.jpg

   MAP_TOP 53.5
   MAP_RIGHT 9.3333333333
   MAP_LEFT 8.6666666667
   MAP_BOTTOM 52.8333333333
\end{verbatim}

The maps can be grouped in sets. You may want to have a map-set for each airfield you fly from.
Another way to use this feature would be to spilt large areas into multiple map sets, if you don't have enough video memory to display all maps at the same time.

Exampls: You have 20 sections for 20 map-tiles in your configuration file.
Now you can put a \texttt{MAP\_CUT} between the first 10 and the second 10
map-tile sections to split into two map-sets.
In ogie you can switch between multiple map-sets by using the "c" and "x" keys.
You may specify more than two map sets by using multiple \texttt{MAP\_CUT}.

Every map-set can be named with \texttt{MAP\_SET\_NAME name} to select it at startup with \texttt{--map-set-name name}, or by its name from the menu.


\paragraph{shifting individual map tiles}
$\,$ \\
\texttt{MAP\_SHIFT\_LAT  degrees} \\
\texttt{MAP\_SHIFT\_LON  degrees} \\
If your maps don't fit exactly, a shift in latitude and/or longitude may be defined.
If  \texttt{MAP\_SHIFT\_...} is given, all following map tiles will be shifted by the given amount, until the shift
is set to zero or to another value.


\paragraph{rgb-format maps}
This shouldn't be used anymore... except you want to use your old maps,
or you don't like lossy compression.
Every map-tile can be in a headerless .rgb (3 byte per pixel) data-format (I
use Image Magick's "convert" to create that format). The size has to
be $2^n$ x $2^n$. That means   you have to scale the image before. Width and height
should be a power of 2 (pixels).
Because the rgb-format is headerless it cannot contain the information about the size
of the image. You need to specify MAP\_WIDTH and MAP\_HEIGHT for each map-tile in your
configuration file.
There is a limit for the maximum pixels for each dimension (width and height).
You can query this limit by executing \texttt{ogie -q}.
Look for \texttt{GL\_MAX\_TEXTURE\_SIZE} [both values (width and height) have to be less or equal
to \texttt{GL\_MAX\_TEXTURE\_SIZE}].


%%% Local Variables:
%%% mode: latex
%%% TeX-master: "GPLIGC_manual.tex"
%%% End:



\subsection{Stereoscopic viewing}

Four stereoscopic modes can be used.
Three of them are \emph{runtime-options} and can be activated by menu.

\paragraph{Double image}
\texttt{--stereo} (or \texttt{STEREO true})
will display 2 stereoscopic images. You can cross the optical axis of your eyes
to get a "real" 3D image (squinting). (Left eye sees right image and vice versa).
Maybe someone will use the "parallel" method (right eye sees right image, and left
one left). Then you should swap the images (press "A").

\paragraph{Anaglyphic modes}
\texttt{--stereo-rg} (or \texttt{STEREO\_RG true})
\texttt{--stereo-rb} (or \texttt{STEREO\_RB true})
For these stereoscopic modes you will need either  red-green or  red-blue
3D-glasses, if left eye is red, you need to swap the images (press "A").

\paragraph{Hardware 3D with shutterglasses}
If you own a  quadro-buffered openGL-card (like nVidia Quadro2, Quadro4...) and
some shutterglasses (or other professional stereo-equipment)
(and the X-server is configured for stereo) you can use the
\texttt{--stereo-hw} option. \emph{OGIE} will try to get a quad-buffered window.
This mode can be initialised at start time only.


\paragraph{Eye distance}
For adjusting the strength of the 3D-effect you can change the distance between
the "virtual" eyes (\texttt{--eye-dist km}, "Q","W" or \texttt{EYE\_DIST value[km]})


\subsection{Projections}
\label{projections}

The flightdata, digital elevation model etc. have to be mapped from earths wgs84 coordinate system to a flat surface. This can be done by using different map projections. \emph{OGIE} offers you four of them.  Which one to be used has to be chosen at start-time of the program. You can use a commandline switch to set the map projection, or you can set a default in the configuration file.
The earth is assumed to be a perfect sphere with a radius of 6371km.

\subsubsection{Projection 1 - cylindric}
\texttt{--projection-cyl-platt} is the commandline switch for this projection. In the configuration file \texttt{PROJECTION 1} can be used. The spheres surface is projected to a cylinder, which is parallel to the axis of the earth and which has the same radius as the sphere. The equator of the sphere is the standard parallel which touches the cylinder. The projection is orthographic.

\subsubsection{Projection 2 - Mercator}
\texttt{--projection-cyl-mercator} is the commandline switch for this projection. In the configuration file \texttt{PROJECTION 2} can be used. This is the well known \emph{Mercator} projection.

\subsubsection{Projection 3 - cylindric}
\texttt{--projection-cyl-no1} is the commandline switch for this projection. In the configuration file \texttt{PROJECTION 3} can be used. This projection is a cylindrical projection, but not geometric. The equator is a standard parallel. The longitude conversion is done like a geometric projection. Latitude is converted in a way, that distances along meridians are preserved.

\subsubsection{Projection 4 - pseudo cylindric}
\texttt{--projection-pseudo-cyl-no1} is the commandline switch for this projection. In the configuration file \texttt{PROJECTION 4} can be used. This is the default projection, which is best suited for small areas. Distances along parallels and meridians are undistorted.



\subsection{Screenshots}
\label{screenshots}
Screenshots can be made using the "p" key for a single shot, or "shift-P" for the continuous screenshot-mode. In the continuous mode every rendered frame is saved. The output format can be specified using the \texttt{--image-format format} option or the configuration file keyword \texttt{IMAGE\_FORMAT format}, where \texttt{format} is one of the following: \texttt{jpg, rgb}.
The names of the image files will start with \emph{frame1000} and the numbers increase.
A different basename can be specified by either \texttt{--basename string} or \texttt{BASENAME string}. Also a path can be given, where to save the screenshots (\texttt{--save-path string} or \texttt{SAVE\_PATH string})

\paragraph{Jpeg}
If the output format is jpeg (which is the default), the jpeg-quality can be set with \texttt{--jpeg-quality int} or in the configuration file \texttt{JPEG\_QUALITY int}, where \texttt{int} is a number between 0 (lowest quality) and 100 (highest quality).

\paragraph{rgb}
While using the rgb format you should keep the information about the image sizes, because this information is not saved within the image. It is a 6-byte per pixel rgb image. You can use \textsc{ImageMagick's} \emph{convert} to convert these into almost every available image format. For example:\\
\texttt{convert -size widhtxheight -depth 16 -endian lsb frame1001.rgb outfile.png}



\subsection{Offscreen rendering}

The offscreen rendering function via GLX is available, but requires GLX 1.3. Offscreen on windows and/or via mesa needs
special compilation... \\

\emph{OGIE} can be used as an offscreen 2D/3D renderer. As in the "onlineplotter", which can be tested on the GPLIGC website. With this function some contest results can be made more visible etc.

Single images can be rendered offscreen. Two modes are available. For image format related options see \ref{screenshots}.

\paragraph{GLX offscreen (pbuffer)}
%\emph{This doesn't work with any Windows! GLX is available on unix/linux only} \\
Offscreen rendering is done using GLX pbuffers and requires GLX 1.3. Rendering is done hardware accelerated, but requires the X-server running and accessible.
Commandlineswitch \texttt{--offscreen} is needed and
a filename for the output can be given by \texttt{--os-outfile filename}

\paragraph{Mesa offscreen (osmesa)}
\emph{This doesn't work with the precompiled binaries! Special compilation is needed}
Commandlineswitch \texttt{--os-mesa} is needed and
a filename for the output can be given by \texttt{--os-outfile filename}
In this mode the rendering is done with the mesa library, but it is hardware independent, no X-Server and no graphics hardware is needed.
For mesa-support \emph{OGIE} has to be compiled as described in section \ref{linux_install}.

\paragraph{Viewpoint}
All other Commandline parameters can be used and the configurationfile will be used.
Important are the \texttt{--init-...} parameters to set the viewpoint and viewdirection.
\texttt{--init-pos-N, W, S, E, NE, SE, SW, NW}  can be used to set the initial position to one
of the borders or corners of the terrain. The view direction will be set to the
centre, if not specified separately (can be used to set the initial position for
the interactive mode too).





\subsection{Performance}

Using the option \texttt{--verbose} (or "VERBOSE true"  in the configfile) will give you
the information how many triangles are used to build the surface. (if DEM is
used) Check by yourself how many triangles you system can handle at a tolerable
speed. The rendering time is also dependent on the quantity of textures used.

In Movie-Mode ("I") with \texttt{--verbose} a framerate is displayed...

A hardware accelerated OpenGL setup is recommended.





\subsection{GPS/Baro alt}

\emph{OGIE} can display the flighttrack based on barometric or GPS altitude:
default behaviour is set in configuration file (GPSALT true|false),
Without configuration file default is barometric.
cmdline-switches \texttt{--baroalt} or \texttt{--gpsalt} can be used.


\subsection{Info}
\label{info}
F6 can be used to switch on/off the info-mode. In infomode the viewpoint position is displayed
at the top left corner of the screen.
In markermode some more information is displayed. The units of speed, vertical speed and altitude can be changed by using factors to convert from standard (km/h, m/s, m) to another unit. These factors can be specified in the configuration file (\texttt{SPEED\_UNIT\_FAC, VSPEED\_UNIT\_FAC, ALT\_UNIT\_FAC}). The names of the units can be set using \texttt{SPEED\_UNIT\_NAME, VSPEED\_UNIT\_NAME} and \texttt{ALT\_UNIT\_NAME}. (See \ref{config}. The timezone can be changed from UTC to localtime using the \texttt{TIME\_ZONE} and \texttt{TIME\_ZONE\_NAME} keyword in the configuration file.


\subsection{Marker}
\label{marker}
F7 activates the "Marker". A huge red arrow pointing to a position of a logged datapoint.
The arrow can be moved forward (F3), backward (F2) and fast forward (F4),
fast backward (F1). If the info-mode is active, some data of the marked
position is displayed.


\subsubsection{Marker-Range}

If your flighttrack crosses the same place several times, you may want a part of
the flighttrack to be displayed only. With \texttt{--marker-ahead n}  and \texttt{--marker-back n}
you specify how many datapoints before and after the marker will be plotted. Default values are
50 back, and 0 ahead. To turn the marker-range-option on, press "shift-U" (\emph{not "u"}), or use
cmdline-switch \texttt{--marker-range}. This may be turned on by default by using
\texttt{MARKER\_RANGE true} in the configuration file (where the range-defaults can be defined
also)

\subsubsection{Follow-mode}

The viewpoint will follow the marker-position. If this option is turned on by
default, can be disabled by cmdline-switch \texttt{--no-follow} or \texttt{FOLLOW false} (in  the configuration file).


\subsubsection{Movie-Mode}
\label{movie}
The Movie-Mode can be switched on by pressing "shift-I", selection from the menu, or with \texttt{--movie}.
The Marker position is continuously moved forward.
Using this mode together with the follow-mode and marker gives a nice movie of the flight.

If it is too fast, you can define a \texttt{MOVIE\_REPEAT\_FACTOR int}, which will render every frame multiple times before shifting the marker. This factor can be changed at runtime with shift-F1 and shift-F2.
\texttt{MOVIE\_REPEAT bool} will switch this on or off. Leaving this off and setting a \texttt{MOVIE\_REPEAT\_FPS\_LIMIT float} will  automatically enable the repeating-mode if a certain framerate is exceeded. \\

\emph{deprecated slow-down method} \\
A default delay can be set in the configurationfile:
\texttt{MOVIE\_TIMER} the argument is in milliseconds. (compiled-in default is 1 msec).
This introduces a `sleep' command, which reduces the responsiveness of the program. Not recommended.

%%% Local Variables:
%%% mode: latex
%%% TeX-master: "GPLIGC_manual.tex"
%%% End: