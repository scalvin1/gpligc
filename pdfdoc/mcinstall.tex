\subsection{Mac OS X}
\label{mac}
This paragraph is mainly written by Matthew Hoover. Also he was the one, who did the testing while gpligc/ogie was ported to Mac OS X.
Owing to his efforts, binaries for ppc- and intel-based Mac OSX machines can be provided.
Thanks a lot, Matthew!

Another section was added by Michael Schlotter, who describes how to install ogie/ogie without using fink.
For that reason Perl/Tk and Gnuplot have to been build from sources. Thanks Michael!


\subsubsection{General}
In order to get gpligc running, you need Perl with the Perl/Tk module. This usually requires to have X11 installed, since Perl/Tk doesn't work with the native Mac OSX GUI (Aqua).
As you also have to install gnuplot, I recommend the easy way shown in section~\ref{fink} using the fink platform.
If you're a more experienced developer (not using fink), you may want to compile Perl/Tk and gnuplot by yourself.
See section~\ref{schlotter}.
To compile the ogie binaries on your machine, you'll need some developer tools (as gcc, make, etc.).
So far, I didn't find a way not to use the jpeglib from fink (if you know, please report).


\subsubsection{Matthew's howto, using fink}
\label{fink}
\begin{enumerate}

\item  Install X11 from the Tools disk included with the OSX package. \\
(X11 is the unix X-Window system and is needed for the  Perl/Tk-stuff (for gpligc)).

\item  Install fink \cite{fink}. \\
Fink is a packaging system that allows Mac users to install, compile and use a wide range of free software.
We need this to install gnuplot and the Perl/Tk-module.

    \begin{enumerate}
        \item  If the user is unfamiliar with fink then also install fink
                commander which is a GUI front end for fink.
        \item  Make sure that the fink or fink commander is able to install
            unstable packages.
            You have to modify \texttt{/sw/etc/fink.conf} (Add main/unstable to the line containing `Trees:')
            More can be found in the fink documentation.
    \end{enumerate}

\item  Install and compile the Perl/Tk module. Therefore you have to check which Perl is installed on your system (\texttt{perl -V}).
    There are different Perl/Tk-module versions (and I am not sure if the newer ones will work with older perls...)
    According to your installed perl choose one of tk-pm560, tk-pm580 or tk-pm581. This example will continue with tk-pm581.\\

Install the tk-pm581 package using fink (\texttt{fink install tk-pm581}) or fink
commander.  There are also tk-pm581-bin and tk-pm581-man.  These will
be installed and compiled and archived when the first is selected.  If
gpligc will not run try unpacking these also.

\item  Install gnuplot using fink (\texttt{fink -b install gnuplot}) or fink commander.  I used the last
stable version with the binary files because it is faster although
there is a newer unstable version.

\item Now you can proceed with the instructions for Unix (see \ref{unix_install}) installation. Be aware to use a binary package for Mac OS X.
The installation paths will be different from those used in Unix/Linux instructions.
For the installation prefix  you can choose by yourself (maybe /sw or /usr).

\item  To run, open X11 and then Terminal.  GPLIGC must be started from
within Terminal and X11 must also be running because it will not be
automagically started.

\item You should add \\
\texttt{test -r /sw/bin/init.sh \&\& . /sw/bin/init.sh} \\
to your \texttt{.bashrc} file to start gpligc in a xterm shell
without a terminal shell opened.

\end{enumerate}


\subsubsection{Michael Schlotter's howto}
\label{schlotter}

\begin{enumerate}
\item  Install Developer Tools from the Mac OSX Installation CD/DVD
\item  Download and install Apples X11-Server
\item  Install X11SDK and BSDSDK. This is done by double-clicking on
    \texttt{X11SDK.pkg} and \texttt{BSDSDK.pkg} in
    \textit{/Applications/Installers/Developer Tools/Packages}
\item Download, build and install gnuplot 4.0.0.
    Don't worry if \textit{make-check} produces some errors.
\item Download, build and install Perl/TK. I used \texttt{Tk-804.027.tar.gz}.
    Don't worry if \textit{make-test} produces some errors.
\item Download and install the latest gpligc with precompiled binaries for Mac
\end{enumerate}
