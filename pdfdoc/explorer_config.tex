%$Id: explorer_config.tex 3 2014-07-31 09:59:20Z kruegerh $
%;;; Local IspellDict: "british"


\section{Configuration file (.ogierc)}
\label{config}

The \texttt{.ogierc} has to be used to set up:


\begin{enumerate}
\item use of a digital elevation model (see \ref{dem})
\item use of digital raster maps (see \ref{maps})
\item use of OpenAir\texttrademark\ airspace file (see \ref{airspace})
\item change the default behaviour of \emph{OGIE}
\end{enumerate}


On  startup of the program  \emph{OGIE} will load a configuration file
\texttt{.ogierc} from your \texttt{HOME} directory.
%On Windows platforms the file is called \texttt{ogie.ini} and is read from the GPLIGC-installation directory.
To use a configuration file from a different location, you can specify the filename by the command line option
\texttt{--config-file FILENAME}.


The \texttt{.ogierc} configuration file may contain keywords-values pairs,
with which the default behaviour can be changed. All keywords \emph{are not} case sensitive.
In the following list of the allowed keywords,  placeholders are used to represent possible values:

\begin{itemize}
\item \texttt{bool} means that one of \texttt{true,on,yes,1,false,off,no,0} (case-insensitive) can be used to turn that option on or off.
\item \texttt{float} stands for a floating point value (such as 54.734 or 0.3)
\item \texttt{integer} should be a integer value (such as 1 or 42)
\item \texttt{file} is to be substituted by a filename with full absolute path (e.g. /usr/share/gpligc/data/dem/demdata.dat or c:\\GPLIGC\\data\\dem\\demdata.dat)
\item \texttt{string} can be any character string (without whitespaces)
\end{itemize}


\subsection{Keywords}
The list of keywords is sorted alphabetically.
If you misspell keywords, you will get a warning in the output. Watch the output after changing your configuration file.

\begin{itemize}

\item \texttt{AIRSPACE bool} \\
Used to set whether airspaces should be displayed by default or not. To define the airspace-describtion file see \texttt{OPEN\_AIR\_FILE}

\item \texttt{AIRSPACE\_x bool} \\
With x one of D, C, CRT, Q, R, P. Used to set whether airspace type x should be displayed by default or not

\item \texttt{AIRSPACE\_LIMIT integer} \\
Upper limit (in FL) for Airspaces. Airspaces with bottom altitude higher than limit, are not shown.

\item \texttt{AIRSPACE\_WIRE bool} \\
This changes the default mode for airspace (wireframe or transparent)

\item \texttt{ALT\_UNIT\_FAC float} \\
Factor to convert from meters to another unit (e.g. 3.28 for feet)

\item \texttt{ALT\_UNIT\_NAME string} \\
Name of the alternative altitude unit

\item \texttt{ANGLE\_OF\_VIEW integer} \\
Angle of view can be a value between 1 and 179 (default=80)

\item \texttt{AUTOREDUCE bool} \\
If the resolution of the requested DEM exceeds the \texttt{MAXTRIANGLES} limit, the upscaling (\texttt{--upscaling}) is reduced and the
downscaling (\texttt{--downscaling}) is increased until the limit isn't exceeded anymore.

\item \texttt{BIGENDIAN bool} \\
Digital elevation data may be present as big endian (most significant byte first, MSB, motorola bytee order) or little endian
(least significant byte first, LSB, intel byte order). The default is yes - big endian, which applies to GTOPO30, SRTM30, SRTM-3 (.HGT),
but \emph{not} to GLOBE and SRTM-3 from seamless-server.

\item \texttt{BASENAME string}\\
The basic filename used to save screenshots can be defined here. It defaults to \emph{frame}.

\item \texttt{BACKGROUND\_COLOR\_[1|2]\_[R|G|B] float} \\
Two background colours can be set. The red, green and blue-value can be set seperately for colour 1 and 2.
The range of the floating point values are limited from 0 to 1.

\item \texttt{BACKGROUND\_STYLE integer} \\
Three background styles are available. 1 correspond to the old style with one solid background (colour 1 is used). A value of 2
will set the background to a vertical gradient from colour 1 (top) to colour 2. The value 3 will switch to a horizontal gradient
(colour 1 is left).

\item \texttt{BORDER float} \\
Width of border in km to be added around the terrain (default=5)

\item \texttt{COLORMAP integer} \\
Colourmap to be used for terrain. See \ref{color}

\item \texttt{COLORMAP\_SEA integer} \\
Colourmap to be used for regions below sea-level. See \ref{color}

\item \texttt{COMPRESSION bool} \\
Whether texture map compression should be used or not (default=no). To use this, the openGL implementation has to support texture map compression (e.g. Mesa does not)

\item \texttt{CURTAIN bool} \\
Whether the blue "curtain" should be drawn or not (default=yes)

\item \texttt{DEM\_COLUMNS integer} \\
Number of columns in digital elevation file

\item \texttt{DEM\_FILE file} \\
Name and path to digital elevation file

\item \texttt{DEM\_GRID\_LAT float} \\
Step width in latitude of digital elevation file

\item \texttt{DEM\_GRID\_LON float} \\
Step width in longitude of digital elevation file

\item \texttt{DEM\_INPUT\_FACTOR float} \\
Scaling factor for DEM data. Should be set in a way, that the result is in meters (e.g. 0.30488 for feet)

\item \texttt{DEM\_LAT\_MAX float} \\
Maximum of latitude in digital elevation file

\item \texttt{DEM\_LAT\_MIN float} \\
Minimum of latitude in digital elevation file

\item \texttt{DEM\_LON\_MAX float} \\
Maximum of longitude in digital elevation file

\item \texttt{DEM\_LON\_MIN float} \\
Minimum of longitude in digital elevation file

\item \texttt{DEM\_ROWS intger} \\
Number of rows in digital elevation file

\item \texttt{EYE\_DIST float} \\
Distance between the "eyes" in stereo viewing modes (in km)

\item \texttt{FGLUT\_CHECK bool} \\
If true, a check for the freeglut version is done. Should be enabled if you use freeglut (to use
some nice freeglut things in future versions). If you use the classic glut, you should left this to
the default (=off) to avoid a harmless warning message.

\item \texttt{FLIGHTSTRIPCOL[UP|DOWN]\_[R|G|B]}\\
Sets one of the colour components Red Green or Blue for the classic flightstrip mode (0). If you want to have the flightstrip displayed in
a single colour, set the colours for \emph{up} and \emph{down} to the same values.

\item \texttt{FLIGHTSTRIP\_LINEWIDTH 2.0}
Set the width of the lines displaying the GPS-track (floating point value in a range of 1.0 -- 7.0).

\item \texttt{FLIGHTSTRIP\_MODE}
Change the default mode for displaying the GPS-track (default=1, 0=classic, 1=altitude, 2=speed, 3=vertical speed).

\item \texttt{FLIGHTSTRIP\_COLORMAP}
Set the default colourmap-type used to display the flight track (integer value, see \ref{color})


\item \texttt{FOLLOW bool} \\
While the "follow-mode" is active, the viewpoint will follow the marker

\item \texttt{FULLSCREEN bool} \\
Whether the OGIE should startup in full-screen mode

\item \texttt{GPSALT bool} \\
Whether the GPS-altitude should be used instead of the barometric altitude

\item \texttt{GRAYSCALE bool} \\
Gray-scale (Black/White) mode (default=off)

\item \texttt{HAZE bool} \\
Atmospheric haze

\item \texttt{HAZE\_DENSITY float} \\
Density of atmospheric haze (0.0 - 0.5) (default=0.01)

\item \texttt{IMAGE\_FORMAT string} \\
This option sets the output format for screenshots. Available options are: \\
\texttt{jpg} jpeg format \\
\texttt{rgb} headerless rgb format (without compression, you need to know width and height to use this image later)

\item \texttt{INFO bool} \\
Info display (shows information about viewpoint and marker-position)

\item \texttt{INFOFONT\_SIZE integer} [ 20-100 ] \\
The default is 40.

\item \texttt{INFOFONT\_LINEWIDTH float} [ 0.5-3.0 ] \\
The default is 1

\item \texttt{INFO\_STYLE integer} [ 1|2 ] \\
1= new style (thanks to \textsc{Antonio Ospite}), 2=old style

\item \texttt{INVERSE\_STEREO bool} \\
Swap right/left image in stereo-modes

\item \texttt{JOYSTICK\_FACTOR\_X float} \\
Scaling factor for joystick-input-value. X-Axis (left-right). Negative values will invert movement. (Default=0.01)

\item \texttt{JOYSTICK\_FACTOR\_Y float} \\
Scaling factor for joystick-input-value. Y-Axis (forward-backward). Negative values will invert movement. (Default=0.01)

\item \texttt{JOYSTICK\_FACTOR\_Z float} \\
Scaling factor for joystick-input-value. Z-Axis (up-down). Negative values will invert movement. (Default=0.01)

\item \texttt{JPEG\_QUALITY int} \\
Sets the quality of the jpeg (0-100, default=75)

\item \texttt{LANDSCAPE bool} \\
Whether terrain should be displayed by default (if digital elevation model is available)

\item \texttt{LIFTS\_COLOR\_[R|G|B] float}\\
If you don't like the default colour of the lifts, you can change it with these keywords.

\item \texttt{MAP bool} \\
Whether textured maps should be displayed (if available)

\item \texttt{MAP\_BOTTOM float} \\
Latitude of lower map border

\item \texttt{MAP\_CUT} \\
Used to separate map-sets

\item \texttt{MAP\_FILE file} \\
Filename of a map-tile (jpeg)

\item \texttt{MAP\_HEIGHT integer} \\
Pixel height of map-tile (not necessary for jpeg)

\item \texttt{MAP\_LEFT float} \\
Longitude of left map border

\item \texttt{MAP\_RIGHT float} \\
Longitude of right map border

\item \texttt{MAP\_SET\_NAME string} \\
Name (identifier) for the map-set

\item \texttt{MAP\_SHIFT\_LAT float} \\
All following map tiles will be shifted in latitude by this value

\item \texttt{MAP\_SHIFT\_LON float} \\
All following map tiles will be shifted in longitude by this value

\item \texttt{MAP\_TOP float} \\
Latitude of upper map border

\item \texttt{MAP\_WIDTH integer} \\
Pixel width of map tile (not needed for jpeg)

\item \texttt{MAPS\_UNLIGHTED bool}
With this set to true, maps will not be lighted when used with DEM and no modulation. (Default is false)

\item \texttt{MARKER bool} \\
Whether the marker should be active by default

\item \texttt{MARKER\_AHEAD integer} \\
How many data points will be displayed (ahead from marker) in marker-range mode

\item \texttt{MARKER\_BACK integer} \\
How many data points will be displayed (backwards from marker) in marker-range mode

\item \texttt{MARKER\_RANGE bool} \\
Marker range mode default

\item \texttt{MARKER\_SIZE float} \\
Marker size (default=1, range=0.01 to 10.0)

\item \texttt{MARKERCOL\_[R|G|B] float}\\
If you don't like the default (red) colour of the maker, you can change it using these keywords.

\item \texttt{MAXTRIANGLES float} \\
The value of maximal allowed triangles for the terrain. If this is exceeded, the terrain is turned
off. (default=1.5e6). To be used in online-plotter applications to avoid Denial-of-Service attacks.
Should be set to a value, which your server can handle in a reasonable time.

\item \texttt{MODULATE bool} \\
Whether the maps should be coloured by digital elevation model  elevation colour scaling

\item \texttt{MOUSE bool} \\
Whether the mouse-pointer is visible or not

\item \texttt{MOVIE bool} \\
If set to true, this will startup ogie in movie mode

\item \texttt{MOVIE\_TIMER integer} (deprecated) \\
Time delay in milliseconds for movie-mode. This reduces the responsiveness of ogie. Better use the next three options.

\item \texttt{MOVIE\_REPEAT bool} \\
Enables multiple rendering of each frame to slow down the marker movement

\item \texttt{MOVIE\_REPEAT\_FACTOR int} \\
Defines how often a frame should be rendered in \texttt{MOVIE\_REPEAT} mode

\item \texttt{MOVIE\_REPEAT\_FPS\_LIMIT float} \\
Set a frame rate limit, above which the \texttt{MOVIE\_REPEAT} mode is activated. When using this option \texttt{MOVIE\_REPEAT}
should be disabled, otherwise it is used at any frame rate.

%\item \texttt{NUMBER\_OF\_MAPS integer} \\
%This option in obsolete. Since 1.3 it is not needed anymore

\item \texttt{OPEN\_AIR\_FILE file} \\
Filename and path of OpenAir\texttrademark  file

\item \texttt{PROJECTION integer} \\
Which projection should be use. See \ref{projections}

\item \texttt{QUADS bool} \\
Use quadrilaterals instead of triangles to build the terrain surface.

\item \texttt{SAVE\_PATH string}\\
The location to store screenshots can be defined by its full path.

\item \texttt{SCALE\_Z float} \\
Scaling factor for z-axis (altitude) (default=3.0)

\item \texttt{SEALEVEL integer} \\
Altitude of sea-level (this is the limiting altitude between colourmap and colourmap-sea)

\item \texttt{SEALEVEL2 integer} \\
If sealevel2 is given, the terrain beneath this value will not be displayed, but covered by sea

\item \texttt{SEALEVEL3 integer} \\
If sealevel3 is given, the terrain beneath this value will be covered by (transparent) sea

\item \texttt{SHADE bool} \\
Usage of goraud shading

\item \texttt{SPEED\_UNIT\_FAC float} \\
Factor applied to the speed. 1.0 is km/h

\item \texttt{SPEED\_UNIT\_NAME string} \\
Name of the speed units

\item \texttt{SPINNING float} \\
This activates spinning around the marker position in movie mode.
In terrain viewer mode this will spin around the centre.
float is the spinning step size in degrees, the sign will decide about the
direction.

\item \texttt{STEREO bool} \\
This activates double image stereo mode

\item \texttt{STEREO\_HW bool} \\
This activates hardware stereo mode (start-time-option)

\item \texttt{STEREO\_RB bool} \\
This activates red/blue anaglyphic stereo mode

\item \texttt{STEREO\_RG bool} \\
This activates red/green anaglyphic stereo mode

\item \texttt{TEXT\_COLOR\_[R|G|B] float}\\
If you don't like the deafault colour of the text (for lifts and waypoints) you can change it with these keywords.

\item \texttt{TEXT\_LINEWIDTH float}\\
This changes the linewidth of the text for lifts and waypoints

\item \texttt{TIME\_ZONE integer} \\
Difference between UTC and your time zone. Do not use the $+$ sign for positive numbers, but $-$ for negative.

\item \texttt{TIME\_ZONE\_NAME string} \\
Name of your time zone

\item \texttt{VERBOSE bool} \\
If this option is active, some output will be made. This option is read if neither --quiet nor --debug are given. (Default=off)

\item \texttt{VSPEED\_UNIT\_FAC float} \\
A factor with which the vertical speed is multiplied. 1.0 is m/s

\item \texttt{VSPEED\_UNIT\_NAME string} \\
Name of the vertical speed units

\item \texttt{WAYPOINTS\_FILE string} \\
Sets the filename (and path) for a waypoint file. See section~\ref{wp} for details.

\item \texttt{WAYPOINTS bool} \\
Determines the default behaviour for drawing waypoints.

\item \texttt{WAYPOINTS\_OFFSET int} \\
draw the text for the waypoints int m higher that their actual elevation. Useful for mountainous terrain. Default=300. If you want the waypoint spheres drawn higher too, you may set WAYPOINTS\_OFFSET\_TEXT\_ONLY to false.

\item \texttt{WAYPOINTS\_OFFSET\_TEXT\_ONLY bool} \\
draws the waypoint spheres using the waypoints offset.

\item \texttt{WP\_COLOR\_[R|G|B] float} \\
Can be used to change the colour of the spheres representing the waypoints.

\item \texttt{WINDOW\_HEIGHT integer} \\
Initial height of the \emph{OGIE} window in pixels

\item \texttt{WINDOW\_WIDTH integer} \\
Initial width of the \emph{OGIE} window in pixels

\item \texttt{WIRE bool} \\
If this option is activated, the surface of the terrain will be drawn as a wire frame model

\end{itemize}

%%% Local Variables:
%%% mode: latex
%%% TeX-master: "GPLIGC_manual.tex"
%%% End:
