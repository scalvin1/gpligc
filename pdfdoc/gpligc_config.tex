\section{The .gpligcrc / gpligc.ini configuration file}
\label{gpligcrc}
The \emph{GPLIGC} configuration file.
The keys and values in this configurationfile are \emph{case-sensitive}.

The following list contains the config-keywords with default values and (for some)
a short description including valid alternative values.

\textbf{Windows}: on windows platforms this file is named \texttt{gpligc.ini}

\begin{itemize}

\item \texttt{DEBUG    "0"} \\
 This should be set to "1" for debugging purposes.

\item \texttt{ENL\_noise\_limit    "500"}
\item \texttt{altitude\_unit\_factor    "1"}
\item \texttt{altitude\_unit\_name    "m"}
\item \texttt{baro\_grid\_large    "1000"} \\
 This changes the spacings of the gridlines.
\item \texttt{baro\_grid\_small    "500"}
\item \texttt{baro\_histo\_intervall    "50"}\\
    Interval for the altitude histograms in m
\item \texttt{browser    "/usr/local/mozilla/mozilla"}\\
 Select your favourite browser here.
\item \texttt{coordinate\_print\_format    "igch"}
\item \texttt{cylinder\_linewidth    "3"}
\item \texttt{distance\_unit\_factor    "1"}
\item \texttt{distance\_unit\_name    "km"}
\item \texttt{draw\_task "1"} \\
defines, whether the task is drawn by default (1) or not (0)
\item \texttt{draw\_wpcyl "1"}\\
defines, whether waypoint cylinders/sectors are drawn by default (1) or not (0)
\item \texttt{fvw\_grid    "yes"}

\item \texttt{fvw\_baro\_grid    "yes"}

\item \texttt{fvw\_baro\_fraction "3"}\\
This determines, how large the barogramm area is compared to the track area.
3 would result in 1/3. (the larger the number, the smaller the barogramm). Allowed values 1.1--10.
Given number does not need to be an integer.

\item \texttt{garmin\_download    "sudo gpspoint -dt -p /dev/ttyUSB2"} \\
this command is used to download tracks from a GPS device.

\item \texttt{geotag\_force\_overwrite    "0"} \\
  by default (value 0) GPS tags in the exifdata are not overwritten. If set to 1, the geotag feature will
overwrite existing GPS tags in images (without further notice).

\item \texttt{gnuplot\_4\_terminal    "0"} \\
 In case of "1" an additional gnuplot shell will be started for each gnu-plot
\item \texttt{gnuplot\_draw\_style    "with lines"}
\item \texttt{gnuplot\_grid\_state    "set grid"}
\item \texttt{gnuplot\_major\_version    "4"}
\item \texttt{gnuplot\_terminal    "x11"}
\item \texttt{gnuplot\_terminal\_app    "xterm -e"}\\
    The terminal application to be used for the gnuplot shell
\item \texttt{gnuplot\_win\_exec    "wgnuplot.exe"}\\
    Contains the filename of the gnuplot-binary on Windows platforms

\item \texttt{gpsbabel\_tdownload  "sudo gpsbabel -t -i garmin -f /dev/ttyUSB2 -o igc -F "}\\
  contains the command string to be used to download trackdata via \emph{gpsbabel}~\cite{gpsbabel}. The last option should be -F, since the output file name will be appended to this string.

\item \texttt{integrate\_over    "10"}\\
    Some of the data plots use integrated values. This will define how many data-points will
    be used for  integration

\item \texttt{map\_max\_tiles "30"}\\ Maximum number of tiles used on the screen (default 30). If this is exceeded the next smaller zoom level will be used.
\item \texttt{map\_max\_scalesize "750"}\\ Maximal dimension for `upscaling` maptiles (default 750).
If map tiles would be scaled beyond this limit, the next higher zoom level is used.
\item \texttt{maps\_zoomlevel "8"}\\ The default map zoom-level (recommended: 8).
\item \texttt{maps "1"}\\ can be 0 or 1. default behaviour maps on/off.
\item \texttt{map\_path "HOME/.gpligc/map"}\\
default directory for maps. Default value depends on the platform.
%Windows: c:\textbackslash GPLIGC\textbackslash map, unix: HOME/.gpligc/map.
If an environment variable GPLIGCHOME is set, a "map" directory will be created within GPLIGCHOME.

\item \texttt{map\_type "osm"}\\  osm=openstreetmap, osmC=openstreetmap Cycle

\item \texttt{marker\_linewidth    "3"}
\item \texttt{mm\_download\_dirs    "Aufnahmen Fotos Videoclips"}\\
	Directories, from where (multimedia) files will be copied (see mm\_mountpoint too)
\item \texttt{mm\_mountpoint    "/mnt/sdC"} \\
	Mountpoint for your multimedia recorder (e.g. mobile phone). Should be user mountable.
\item \texttt{mm\_player    "mplayer"} \\
	Player to be used for multimedia files (audio recordings, movies, etc)

\item \texttt{new\_version\_message\_shown    "0.1"}

\item \texttt{open\_additional\_info    "0"}\\
    If set to 1, the `additional info dialog' is opened immediately in cases where no gpi-file is found.

\item \texttt{optimizer\_cycles\_mmc    "20"} \\
	For all of the \texttt{optimizer\_*} keys see section~\ref{optimise}
\item \texttt{optimizer\_cycles\_sa    "5"}
\item \texttt{optimizer\_debug    "0"}
\item \texttt{optimizer\_method    "mmc"}
\item \texttt{optimizer\_mmc    " -m 1000 -mmc 25000 -devisor 3 -refine 2 "}
\item \texttt{optimizer\_sa    " -sima -m 1000 -sacycles 500 -saexp -sapara 15.0 -saparb 0.03 -devisor 3 -refine 2 "}
\item \texttt{optimizer\_verbose    "0"}


\item \texttt{photo\_path    "none"}
\item \texttt{photos    "1"}
\item \texttt{picture\_viewer    "internal"}\\
    Whether to use the "internal" or any other picture viewer. My favourite is "kuickshow"
\item \texttt{skip\_check    "1"}\\
    With "1" a skip-check is performed. If the difference between to logged positions is larger than
    skip\_limit\_minutes the skip will be marked.
\item \texttt{skip\_del\_first\_after    "1"} \\
	This circumvents a bug in the Garmin Geko tracklogs. The first position fix after a skip will be discarded.
\item \texttt{skip\_limit\_minutes    "0.2"}\\
    Limit to detect skips in the tracklog
\item \texttt{speed\_histo\_intervall    "5"}\\
    Interval for the speed histogram in km/h
\item \texttt{speed\_unit\_factor    "1"}
\item \texttt{speed\_unit\_name    "km/h"}

\item \texttt{starting\_line    "10"}\\
  Length of the starting line in km. The line is displayed with the FAI-sectors for the first WP.


\item \texttt{task\_linewidth    "3"}
\item \texttt{terminal    "xterm -hold -e"} \\
	terminal application to be used for some things (copying Multimedia files, downloading from Garmin)

\item \texttt{te\_vario\_fallback "0"}\\
The total energy compensated vario is usually calculated from the airspeed. Since many loggers dont log airspeed,
the total energy compensation can be calculated from groundspeed (errors can be large in case of significant wind).

\item \texttt{te\_warning    "1"}\\
A warning on groundspeed total energy compensation can be disabled setting this to 0.

\item \texttt{terminal	"xterm -hold -e"}\\
Terminal command used for download of garmin-tracks and media

\item \texttt{timezone    "0"}\\
Offset to local timezone (e.g. timezone used in your camera) . Used for the photo locator.

\item \texttt{vario\_histo\_intervall    "0.5"}\\
    Interval for the vertical speed histograms in m/s
\item \texttt{vertical\_speed\_unit\_factor    "1"}
\item \texttt{vertical\_speed\_unit\_name    "m/s"}

\item \texttt{viewclick\_res "1"}\\
	If you experience serious delays in moving the crossmarks by clicking close to the track, you may increase this number to a larger integer

\item \texttt{waypoint\_linewidth    "3"}
\item \texttt{wind\_analysis "1"} \\
If set to "1", an airspeed-groundspeed difference is plottet with F5/F6/F7 statistics.

\item \texttt{window\_height "500"} \\
\item \texttt{window\_width "900"} \\
size of the plotting area of the main window (pixels)


\item \texttt{working\_directory    "/home/user1/IGC"}
\item \texttt{zoom\_sidelength    "10"}\\
    Sidelength of the zoom-window in km

\item \texttt{zoom\_border    "3"}\\
    Border in km to add around flight track.

\item \texttt{zylinder\_names    "1"}\\
 Display waypoint names next to cylinders (0=no, 1=yes).
 
\item \texttt{zylinder\_radius    "0.5"}\\
 Radius (in km) of the waypoint cylinders (aka \emph{barrels}).

\item \texttt{zylinder\_wp\_type    "both"}

\end{itemize}
