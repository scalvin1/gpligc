%$Id: bugs.tex 3 2014-07-31 09:59:20Z kruegerh $
%spellch 1.11
%;;; Local IspellDict: "british"


\section{Known Bugs}
\label{bugs}

% install.sh discontinued.
% \subsection*{Errors / Syntax errors in install.sh}
% If you get a message like
% \begin{verbatim}
% ./install.sh: 40: Syntax error: "do" unexpected (expecting "}")
% \end{verbatim}
% after starting the ./install.sh script, you're using a shell, which is not compatible to the bash shell.
% \texttt{install.sh} needs to be run through a bash compatible shell. Try something like \\
% \texttt{sudo /bin/bash ./install.sh} \\
% or \\
% \texttt{sudo /bin/zsh ./install.sh} \\
% Shells, known to work: \texttt{bash, ksh, zsh} \\
% Shells, known not to work: \texttt{csh, dash}

\subsection*{Menu in OGIE (freeglut)}
With older freeglut versions (2.4.0) some problems arise due to the positioning and the size of the
menu, which sometimes is placed in a way, that parts of it don't fit the screen.
In this case it's possible to release the mouse pointer (key m --- also see next bug, sorry) and right-click in a different position. That way it's possible to reach all entries from the menu.
In freeglut 2.6.0 this seems to be solved.

\subsection*{Crash on activating the mousepointer (freeglut 2.4.0)}
Ogie will crash, if the mousepointer is activated by key m. This is a known bug in freeglut.
There is a patch for this in freeglut-cvs. On Gentoo systems use freeglut-2.4.0-r1 or later.

\subsection*{Menu does not work in OGIE (freeglut 2.8.0)}
Due to a bug in freeglut 2.8.0 the menu does not work at all.
Please update to freeglut 2.8.1


%\subsection*{ogie zombi process on windows after using exit window button}
%If you quit ogie by using the exit (x) button from the window decoration,
%ogie may end up as a zombie process, consuming lots of cpu time.
%Use the ESC key, or quit from the menu to end ogie.


\subsection*{Weird speedogram plot}
If your (very old) perl-distribution provides the module Math::Complex earlier
than version 1.26, some errors will occur. Most likely negative values in
speedogram! This is a bug in the old Math::Complex perl module.


\subsection*{Annoying open dialog}
The "open File" dialog starts always in "/". This is  a bug in the
Tk module. To avoid that, update to Perl/Tk Module "Tk-800.024" or later.
This is a bug in the Perl/Tk module before 800.024


%\subsection*{Warnings on Windows}
%This bug shouldn't occur in GPLIGC 1.2 and later (avoided by a workaround). \\
%\texttt{Use of uninitialised value in concatenation (.) or string at
%\\ C:$\backslash$Perl$\backslash$site$\backslash$lib$\backslash$Tk.pm line 350}  (or line 361).
%If you get two of these messages, after starting GPLIGC you have an older Perl/Tk module.
%(800.023 and later shouldn't be affected). Nothing to worry about. Just ignore.
%It is related to the HOME environment variable not being set (by default)
%under Windows. Possible workaround set the environment variable HOME, by putting following
%line to your autoexec.bat: \\
%\texttt{SET HOME=C:$\backslash$}\\
%(Better: update GPLIGC to most recent version).


%\subsection*{Red/Green Blue/Red stereo viewing modes do not work on Windows 98}
%On my win98 (SE) test-system the anaglyphic stereo viewing modes doesn't work. These modes result in black and white rendering. I'm not sure if this is a bug in older MS openGL implementation or in the specific hardware drivers. If you have similar problems, please report your specific operating system and hardware.



%%% Local Variables:
%%% mode: latex
%%% TeX-master: "GPLIGC_manual.tex"
%%% End:
