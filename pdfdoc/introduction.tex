%$Id: introduction.tex 3 2014-07-31 09:59:20Z kruegerh $
%;;; Local IspellDict: "british"
%spellchecked 1.16-1.17

\section{Introduction}
\emph{GPLIGC} is a software package for glider pilots, hang- and paraglider pilots, and for all others,
who want to analyse and visualise GPS track logs.
\emph{GPLIGC} reads track logs from files in igc-format as specified by the International Gliding Commission \cite{igc}.
Extracting the data from the GPS devices and conversion to the igc format has to be done with third-party software.
(GPS tracks can be downloaded from some Garmin devices using \emph{gpspoint}~\cite{gpspoint},
Nokia/Symbian mobile phones can be used as loggers utilising \emph{gsil}~\cite{gsil}
and another option is to use \emph{gpsbabel} \cite{gpsbabel}).

The package contains two main programs: (1) \emph{gpligc}, analysation and (2) \emph{ogie},
3D visualisation (can also be used as a digital elevation data viewer).

The software can be used under the terms of the GNU General Public License (see appendix \ref{gpl}),
which means that it's free and the source code is available.
For details read the license, which is included in appendix~\ref{gpl}.

The webpage of \emph{gpligc} can be found at \cite{gpligc}.

\subsection{GPLIGC}
\emph{GPLIGC} is a flight data analysing software. Its name is assembled from \textbf{GPL} (the GNU General Public License, \cite{fsf}), \textbf{G}nuplot (free plotting software, \cite{gnuplot}), \textbf{P}erl (the famous scripting and programming language, \cite{perl}), \textbf{L}ogger (flight data recorder) and \textbf{IGC} (the International Gliding Commission and name of the flight data file format, \cite{igc}).
\emph{GPLIGC} is written in Perl \cite{perl}, using the Perl/Tk module \cite{perltk} for the graphical user interface.
Track and altitude plots can be visualised in a simple way and some basic statistical information can be calculated.
The recorded data can be analysed in detail.
Optimisation for the onlinecontest can be performed. Turn-point observation zones can be displayed.
\emph{Gnuplot} \cite{gnuplot} is used to generate some plots (barogram, GPS-altitude, vertical speed, speed, noise level, etc.) of the data either to the screen or some graphical file format (including png, fig, ps, eps).
\emph{GPLIGC} is able to locate coordinates of photos, which have been taken with a digital camera, while logging GPS data.
To use this geo-tagging feature a correct timestamp in the JPEGs EXIF header is needed or it should be retained as the files timestamp.

The development of \emph{gpligc} started in January 2000.


\subsection{OGIE}
\emph{OGIE} is a  program written in C++ using OpenGL and GLUT (or freeglut \cite{freeglut}) libraries.
The flight data can be visualised in 3D (even in \emph{real 3D}, using stereoscopic methods).
The viewpoint can be controlled in several  ways (egocentric, swivel/rotate or coupled with the flight).
Digital elevation models can be used to display the terrain, digitised maps can be used, and airspaces from OpenAir\texttrademark-files can also be displayed.
Colour scaling can be applied to the terrain data, the digitised maps and to the flight-track itself.
\emph{OGIE} can also be used as a digital elevation model viewer.
\emph{OGIE} is able to render offscreen. Images can be generated hardware accelerated, or hardware independent (with Mesa \cite{mesa}). This can be used to generate images for contests etc. (server use).
\emph{OGIE}s name was assembled from \emph{\textbf{o}pen\textbf{G}L\textbf{I}GC\textbf{e}xplorer}: \textbf{openGL} (the open Graphics Library), \textbf{IGC} \cite{igc}, \textbf{explorer}.


The development of \emph{ogie} started in 2002. Until 2010 the long name \emph{openGLIGCexplorer} was used.


\paragraph{How they work together}
Basically \emph{gpligc} and \emph{ogie} are independent pieces of software.
\emph{OGIE} was designed to be an independent 3D visualisation-only tool, because Perl is too slow for that task.
However, if you start \emph{ogie} from within \emph{gpligc} some data (altitude calibration data, marked lifts, etc.) will be put forward to \emph{ogie}.
%But if you start the \emph{openGLIGCexplorer} by the button in \emph{GPLIGC}, there will be some communication between them...

\subsection{Contact, bug reports, feature requests}
Bug reports and feature requests should be submitted via the \emph{gpligc} support page at Sourceforge \cite{gpligc}.
I recommend to sign up for the \emph{gpligc-announce} mailing list \cite{gpligc}, which I use to inform users of updates or serious bugs, etc. (very low traffic)




%%% Local Variables:
%%% mode: latex
%%% TeX-master: "GPLIGC_manual.tex"
%%% End:
