\subsection{Waypoints}
\label{wp}

Waypoints can be displayed by ogie.
The file containing the waypoints can be declared in the ogie-configuration file by the keyword \texttt{WAYPOINTS\_FILE} or by a commandline argument \texttt{--waypoints-file}. Some more keywords and command-line arguments are available to change the default behaviour.

To switch them on or off use the F12 key. Page-up and page-down can be used to change the size of the spheres, the text-size can be changed with shift-page-up/down. Using shift-pos1 or shift-end changes the displayed text (waypoint-long name, waypoint short-name, waypoint-altitude, waypoint-symbol name).

\subsubsection{Format of the waypoint file}
As there are probably hundreds of waypoint file formats available, I chose a simple one, which I use with my handheld Garmin GPS and \emph{gpsbabel} \cite{gpsbabel}.
It has six columns of data: latitude (degrees), longitude (degrees), altitude (metres), short name (max six letters), long name, symbol name.
Columns are seperated by whitespaces (therefore no whitespaces are allowed within the names).
Using \emph{gpsbabel} it should be easy to convert any format to this.
You'll find a gpsbabel-style (\texttt{gpligcwpt.gpsbabelstyle}) file in the PREFIX/share/gpligc folder.
Here is an example how to convert a cambridge waypoint file to the needed format:\\

\texttt{gpsbabel -i cambridge -f cambridgefile  -o xcsv,style=gpligcwpt.gpsbabelstyle -F mywpts.gwpt}\\
The important part is the output format option `xcsv,style=' using the provided style file.\\
However, \emph{gpsbabel} is cool, you should have a look at it anyway.
You can even download your waypoints from a Garmin device like this:\\
\texttt{gpsbabel -i garmin -f /dev/ttyS1 -o xcvs,style=gpligcwpt.gpsbabelstyle -F outfile.gwpt}
