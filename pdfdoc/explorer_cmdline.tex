\section{Commandline options (OGIE)}
\label{cmdline}

\emph{OGIE} has a lot of commandline options. For a full list try:\\
\texttt{ogie --help} \\
Commandline arguments override configuration file settings, which override
compiled-in-defaults. For most options there is an option to turn the feature on, and another option to turn it off, so you are able to override all
possible configuration file settings. 
Many of the options can be changed at runtime via keyboard input or menus, but some are \emph{start-time-options}, which can be activated or set at the time the program is started, only.
These are marked (ST).

Options marked as \emph{default} are active by default (compiled in). This may be overridden by the configuration-file, or commandline switches.

Some of the commandline-switches (or commandline-options) require  additional parameters (\texttt{INT, FLOAT, FILE} or \texttt{STRING}), the meaning of these can be read in section \ref{config}.

\subsection{All available commandline options}

\begin{itemize}

\item \texttt{-h, --help} \\
Print help and all available commandline options (ST)

\item \texttt{-V, --version} \\
Print version information (ST)

\item \texttt{-v, --verbose} \\
Be verbose, lots of console output (ST)

\item \texttt{--quiet} \\
Turn off verbosity no output to console. Overrides --verbose and VERBOSE (ST)

\item \texttt{-q, --query-gl} \\
Querying openGL implementation. This will give out some information about your specific OpenGL implementation (ST)

\item \texttt{--check} \\
This returns exitcode 0. Used by GPLIGC to check if OGIE is available (ST)

\item \texttt{--debug} \\
This overrides --verbose and --quiet. Lots of ugly debugging output (ST).

\item \texttt{--compiler} \\
This will give out some information about compiler and building environment (ST)

\item \texttt{-i FILE, --igc-file=FILE} \\
This option specifies the igc-file to be opened (ST)

\item \texttt{--gpsd} \\
Try to connect to local gpsd, retrieve position data, and start in live-mode (ST).
Ogie has to be compiled with gpsd support (See section \ref{unix_install}).

\item \texttt{--gpsd-server=STRING} \\
network address of gpsd server (ST)

\item \texttt{--gpsd-port=INT} \\
port number of gpsd server (ST)


\item \texttt{-g, --gpsalt} \\
The altitude from GPS will be used instead of barometric (ST)

\item \texttt{-b, --baroalt} \\
Barometric altitude will be used (default, ST)

\item \texttt{--use-all-fixes} \\
Use all position fixes. Even those which are flagged invalid. (ST)

\item \texttt{--lat=FLOAT} \\
Latitude of centre. Used for terrain viewing without igc-file (ST)

\item \texttt{--lon=FLOAT} \\
Longitude of centre. Used for terrain viewing without igc-file (ST)

\item \texttt{--get-elevation} \\
To be used with --lat and --lon. Will return the elevation of the given coordinates,
if a DEM is configured. To be used with SRTM-3 data. (elevation=0 is the void-flag, at least in the usgs seamless server downloads).
INVALIDn is returned, if n neighbouring grid-points are invalid. n=9 is returned, if the requested
position is not covered by the configured DEM.
The second value which is returned is the max difference in elevation of the four neighbouring grid-points, the maximum of the remaining
neighbours for INVALID1-3, 0 for INVALID4 and 9999 for INVALID9. (ST)


\item \texttt{-c FILE, --config-file=FILE} \\
Used to open a non-standard config file (ST)

\item \texttt{-o, --ortho} \\
Forces startup in 2D orthographic viewing mode

\item \texttt{--perspective} \\
Forces startup in  3D viewing mode

\item \texttt{--aov=INT} \\
This sets the angle of view (1-179)

\item \texttt{-l, --landscape} \\
Use digital elevation data to display terrain

\item \texttt{-f, --flat} \\
Don't use terrain from DEM. Use flat surface instead

\item \texttt{-m, --map} \\
Activates displaying of digitised maps. If configured in configuration file.

\item \texttt{--no-map} \\
Don't use digitised maps

\item \texttt{--map-set-name=STRING} \\
Name of map set to use as default

\item \texttt{--modulate} \\
Maps are coloured by DEM altitude colour, if this option is active

\item \texttt{--no-modulate} \\
Use original colour of maps

\item \texttt{--maps-unlighted} \\
If maps used with DEM and \texttt{--no-modulate}, this turns off lighting of the maps (ST)

\item \texttt{--maps-lighted} \\
Maps are lighted, if used with DEM and no modulation. (Default behaviour) (ST)

\item \texttt{--no-lighting} \\
Don't use lighting. Use for orthomode with upscaling recommended (ST)

\item \texttt{--terrain-shading} \\
Terrain shading. This implies the \texttt{--no-lighting} option (ST)

\item \texttt{--shading-scale=FLOAT} \\
Strength of terrain shading. The smaller the value, the stronger the effect.
If not given, the max elevation difference divided by seven is used (ST)

\item \texttt{--light-direction=INT} \\
The direction of the light for terrain shading. 1 corresponds to north, 2 nort-east, 3 east, 4 south-east, 5 south,
6 south-west, 7 west, 8 north-west (ST)

\item \texttt{-a, --airspace} \\
Turn on airspace visualisation

\item \texttt{--no-airspace} \\
Turn off airspace visualisation

\item \texttt{--airspace-wire} \\
Turn on airspaces in wireframe mode

\item \texttt{--airspace-wire-col-[r|g|b]} \\
Defines colours for airspace wireframe lines

\item \texttt{--airspace-wire-width}\\
Sets the linewidth used to draw airspace wireframes.

\item \texttt{--airspace-transparent} \\
Turn on transparent airspace.

\item \texttt{--airspace-limit=INT} \\
Airspaces, which lower boundary is higher that this limit (in FL), will not be shown. (ST)

\item \texttt{--airspace-file=FILE} \\
Use airspaces from file (OpenAir\texttrademark -format) (ST)

\item \texttt{-w, --wire} \\
Draw terrain surface as wireframe-model

\item \texttt{--filled} \\
Use filled polygons for terrain (default)

\item \texttt{--grayscale} \\
Use gray scaled image

\item \texttt{--color} \\
Use coloured image (default)

\item \texttt{--stereo} \\
Use double image stereoscopic mode

\item \texttt{--no-stereo} \\
Do not use stereoscopic modes (default)

\item \texttt{--stereo-rg} \\
Use anaglyphic stereoscopic mode (red/green)

\item \texttt{--no-stereo-rg} \\
Do not use anaglyphic stereoscopic mode red/green (default)

\item \texttt{--stereo-rb} \\
Use anaglyphic stereoscopic red/blue mode

\item \texttt{--no-stereo-rb} \\
Do not use anaglyphic stereoscopic red/blue (default)

\item \texttt{--stereo-hw} \\
Use stereoscopic hardware if available (ST)

\item \texttt{--no-stereo-hw} \\
Do not use stereoscopic hardware (default)

\item \texttt{--inverse-stereo} \\
Swap right/left image for stereoscopic modes

\item \texttt{--no-inverse-stereo} \\
Don't swap images right/left (default)

\item \texttt{--eye-dist=FLOAT} \\
Set eye distance for stereoscopic viewing modes (default=0.2km)

\item \texttt{--flat-shading} \\
Do not use gouraud shading (every triangle of the surface will get the same colour)

\item \texttt{--gouraud-shading} \\
Use gouraud-shading (default)

\item \texttt{--quads}\\
Use quadrilaterals to build terrain surface (ST)

\item \texttt{--curtain} \\
Draw curtain (default)

\item \texttt{--no-curtain} \\
Do not draw curtain

\item \texttt{--haze} \\
Enable atmospheric haze

\item \texttt{--no-haze} \\
Do not use atmospheric  haze (default)

\item \texttt{--haze-density=FLOAT} \\
haze density (0.0 clear - 0.5 dense fog)

\item \texttt{--colormap=INT} \\
Use colourmap \texttt{INT} for terrain surface (see \ref{color})

\item \texttt{--colormap-sea=INT} \\
Use colourmap \texttt{INT} for seafloor (see \ref{color})

\item \texttt{--colormap-min=INT} \\
Minimum altitude for colour scale (ST)

\item \texttt{--colormap-max=INT} \\
Maximum altitude for colour scale (ST)

\item \texttt{--sealevel=INT} \\
Elevation of sealevel. Beneath this elevation seafloor colourmap is used, above terrain colourmap (ST)

\item \texttt{--sealevel2=INT} \\
Elevation of sealevel2. The blue ocean will be drawn at elevation of sealevel2 (ST)

\item \texttt{--sealevel3=INT} \\
Elevation of sealevel3. A transparent blue surface will be drawn at elevation of sealevel3 (ST)

\item \texttt{--ignore-elev-[min,max]=INT} \\
Defines limits or a range, which are not used for determining the extrem values of the topography (ST)

\item \texttt{-s FLOAT, --scalez=FLOAT} \\
Z-axis scaling. A factor of 1.0 represents the \emph{real} relations. A default value of 3.0 is used to emphasise altitude

\item \texttt{-d INT, --downscaling=INT} \\
DEM raster downscaling can be used to reduce resolution of surface (to show larger areas) (ST)

\item \texttt{--upscaling INT}
The resolution of the DEM raster is enhanced by interpolation. Use with care,  higher factors increase the number of
triangles used dramatically. Good for small area terrain display (ST)

\item \texttt{--fullscreen} \\
Start up in fullscreen mode

\item \texttt{--window} \\
Start windowed (not fullscreen, default)

\item \texttt{--width=INT} \\
Set initial width of window (pixels)

\item \texttt{--height=INT } \\
Set initial height of window (pixels)

\item \texttt{--border=FLOAT } \\
Adds a \texttt{FLOAT} km border at top, bottom, right and left margin (ST)

\item \texttt{--border-lat=FLOAT} \\
Adds a \texttt{FLOAT} km border at top and bottom margin (ST)

\item \texttt{--border-lon=FLOAT} \\
Adds a \texttt{FLOAT} km border at right and left margin (ST)

\item \texttt{--offset=INT} \\
Shifts the flight \texttt{INT} meters up (relative to the dem surface)

\item \texttt{-e INT, --airfield-elevation=INT} \\
Sets the elevation of the take-off location (in m). The relative shift of the flight will be calculated automatically

\item \texttt{--marker-pos=INT } \\
Set the position of the marker to datapoint number \texttt{INT}

\item \texttt{--marker-time=string } \\
Set the position of the marker to datapoint nearest to HH:MM:SS

\item \texttt{--marker } \\
Activated the marker at start-time

\item \texttt{--marker-size=FLOAT}\\
Size of the Marker (0.01-10)

\item \texttt{--no-marker} \\
Disables marker at start-time (default)

\item \texttt{--info} \\
Activates info text display at start-time

\item \texttt{--no-info} \\
Turns the info text display off (default)

\item \texttt{--text=STRING} \\
With this option a text string can be specified, which will be displayed
in the first line of the info text (ST)

\item \texttt{--no-position-info}\\
This option removes the information about the viewpoint position.

\item \texttt{--no-marker-pos-info}\\
To turn off the information about the marker position use this option.

\item \texttt{--text-size=FLOAT}\\
Size of text for points/lifts (0.001-1.0)

\item \texttt{--text-width=FLOAT}\\Width of text (1-20)

\item \texttt{--lifts=STRING}\\GPLIGC liftsfile (ST)

\item \texttt{--lifts-info-mode=INT}\\which info to display (1= int. vertical speed, 2=verical speed, 3=altitude, 4=time, 5=time, 6=date, 7=file)


\item \texttt{--waypoints-file=STRING}\\
  set the waypointsfile (see~\ref{wp}). Overrides any names given in the config file
\item \texttt{--waypoints}\\
	show waypoints  (default=off)
\item \texttt{--no-waypoints}\\
        dont show waypoints  (default=off)
\item \texttt{--waypoints-info-mode=INT}\\
	sets info to display (1-4: 1=description, 2=name, 3=altitude [m], 4=symbol).

\item \texttt{--waypoints-offset=INT}\\
        draw the text for the waypoints INT m higher that their actual elevation. Useful mountenous terrain. Default=300. If you want the waypoint spheres drawn higher too, you may set WAYPOINTS\_OFFSET\_TEXT\_ONLY to false (see~\ref{config}) or use \texttt{--waypoints-offset-spheres}.

\item \texttt{--waypoints-offset-spheres=INT}\\
        draw the text and spheres for the waypoints INT m higher that their actual elevation. Useful mountenous terrain.

\item \texttt{--flighttrack-mode=INT}\\
Sets the mode of track display. The interger value can be one of 0,1,2 or 3. 0: Classic mode, two colours (climbing, descending), colours can
be changed with \texttt{FLIGHTSTRIPCOL[UP|DOWN]\_[R|G|B]}. 1: Colour gradient (see \texttt{--flighttrack-colormap}) is used to display altitude
(this is the default).
2: Colour gradient is used to show speed. 3: Colour ramp is used to show vertical speed.

\item \texttt{--flighttrack-colormap=INT}\\
Sets the colourmap used to display the flighttrack. Integers from 1 to 7 can be used (see \ref{color}).

\item \texttt{--flighttrack-linewidth=FLOAT}\\
Sets the linewidth of the flighttrack. Floating point values in the range of 1.0 -- 5.0 can be used.

\item \texttt{--follow} \\
The viewpoint will be coupled with marker (default)

\item \texttt{--no-follow} \\
Makes viewpoint independent of marker position

\item \texttt{--marker-range} \\
A range (in time; future-past) around marker is plotted only

\item \texttt{--no-marker-range} \\
Full flight data is displayed (default)

\item \texttt{--marker-ahead=INT} \\
Defines marker range (datapoints in future of marker position) (ST)

\item \texttt{--marker-back=INT} \\
Defines marker range (datapoints in the past of marker position) (ST)

\item \texttt{--movie} \\
This will start up ogie in movie mode.

\item \texttt{--cycles=INT} \\
If given, ogie will perform INT cycles in movie mode before exiting (ST)

\item \texttt{--spinning=float} \\
Whether ogie should do spinning in movie mode. float is an angular value (in degrees),
its sign determines the direction of spinning (ST)


\item \texttt{--smooth-mouse} \\
Mouse movement will be damped (ST)

\item \texttt{--parent-pid=INT} \\
PID of parent process. To this PID the signal SIGUSR1 will be send on exit

\item \texttt{--compression} \\
Use texture map compression

\item \texttt{--no-compression} \\
Do not use texture map compression (default)

\item \texttt{--offscreen} \\
Render a single image and output to jpeg. Offscreen rendering with GLX

\item \texttt{--osmesa} \\
Render a single image and output to jpeg. Offscreen with Mesa

\item \texttt{--os-outfile=FILE} \\
Sets filename for offscreen rendered jpeg-image (ST)

\item \texttt{--jpeg-quality=INT} \\
Sets jpeg Quality (compression level, 0-100) of jpeg output (ST)

\item \texttt{--image-format=STRING} \\
Sets output format for screenshots. Available jpg, rgb (ST)

\item \texttt{--save-path=STRING}\\
Via this option location for saving screenshots can be given. (ST)

\item \texttt{--basename=STRING}\\
The basename of screenshots can be defined. A number and file extension will be added automatically (ST)

\item \texttt{--clipping-far=FLOAT}\\ (ST)
\item \texttt{--clipping-near=FLOAT}\\
Distance of the clipping planes in km. Default is 0.2 or 1.0 (near) and 600 or 1000 (far). Change this if you need.
This influences depth-buffer accuracy! (ST)

\item \texttt{--init-lat=FLOAT} \\
Sets latitude of initial viewpoint position (ST)

\item \texttt{--init-lon=FLOAT} \\
Sets longitude of initial viewpoint position (ST)

\item \texttt{--init-alt=INT } \\
Sets altitude of initial viewpoint position (ST)

\item \texttt{--init-heading=INT } \\
Sets initial viewing direction (heading in degree) (ST)

\item \texttt{--init-dive=INT } \\
Sets initial viewing dive angle (degrees downwards from horizontal) (ST)

\item \texttt{--init-pos-N} \\
Sets initial position above north border of scene (ST)

\item \texttt{--init-pos-E} \\
Sets initial position above east border of scene (ST)

\item \texttt{--init-pos-S} \\
Sets initial position above south border of scene (default) (ST)

\item \texttt{--init-pos-W} \\
Sets initial position above west border of scene (ST)

\item \texttt{--init-pos-NE} \\
Sets initial position above north-east corner of scene (ST)

\item \texttt{--init-pos-SE} \\
Sets initial position above south-east corner of scene (ST)

\item \texttt{--init-pos-SW} \\
Sets initial position above south-west corner of scene (ST)

\item \texttt{--init-pos-NW} \\
Sets initial position above north-west corner of scene (ST)

\item \texttt{--init-pos-center} \\
Sets initial position in the centre of the scene (ST)

\item \texttt{--init-ortho-lat=FLOAT} \\
Sets initial latitude of orthographic viewing mode centre (ST)

\item \texttt{--init-ortho-lon=FLOAT} \\
Sets initial longitude of orthographic viewing mode centre (ST)

\item \texttt{--init-ortho-width=FLOAT} \\
Sets initial width of orthographic-viewing [km] (ST)

\item \texttt{--projection-cyl-platt} \\
Sets `platt' projection (ST)

\item \texttt{--projection-cyl-no1} \\
Sets cylindric projection 1 (ST)

\item \texttt{--projection-pseudo-cyl-no1} \\
Sets pseudocylindric projection 1 (ST, default)

\item \texttt{--projection-cyl-mercator} \\
Sets Mercator projection (ST)

\end{itemize}
