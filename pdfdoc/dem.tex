\subsection{Digital Elevation Model}
\label{dem}

There are many digital elevation models on the web, which can be downloaded for free and used with ogie:
ETOPO2 \cite{etopo2}, GLOBE \cite{globe}, GTOPO30 \cite{gtopo30}, SRTM30 Plus (TOPO30) \cite{srtm30plus}, SRTM30 \cite{srtmv2}, SRTM-3 \cite{srtmv2} and SRTM-1 \cite{srtmv2}.
GTOPO30, SRTM30 (Plus) and GLOBE have a resolution of 30 arc-seconds, 1km.
ETOPO2 has a 2-minute grid (4km), but also covers the oceans.
SRTM-3 is 3 arc-seconds (90m) and SRTM-1 (only available for U.S.) is 1 arc-second (30m).

The Shuttle Radar Topography Mission (SRTM) topographic data
with resolution 1 arc-second (for USA) and 3 arc-second for almost the rest of the world
is available, but due to the high resolution not very good for regular flight-analysis. (Graphics
hardware won't handle larger areas).

My recommendation is to use either GTOPO30 or SRTM30 (Plus).
If you have lots of hard disk space and want to analyse flights from many countries, you should consider
to build a \texttt{WORLD.DEM} from GTOPO30/SRTM30. If you like to explore the oceans, you may merge
it with ETOPO2 data. Data of this type is available from the gpligc download directories for many countries (including needed configuration settings).

SRTM-3 and SRTM-1 can be used for small-scale high-resolution application.


\subsubsection*{Data format}

Binary data in 2 byte integer (big endian byte) format is needed.
(You can get these directly from GLOBE, GTOPO30 and ETOPO2 Web sites)
Little endian data can be used with the config option: \texttt{BIGENDIAN false}.


\subsubsection{GTOPO30, SRTM30}
The worldwide GTOPO30 \cite{gtopo30} elevation model is split up in 33 pieces (tiles).
Get the tile you need and put the full path to the *.DEM file into the
configuration-file (see \ref{demconf}). You also need to set the rows and columns
and minima and maxima and grid resolution.

If you have lots of space on your hard-disk and a fast internet connection you should consider to
get all (33) tiles (about 280~MB compressed) and use the \texttt{createworld}
tool to generate a \texttt{WORLD.DEM} (single file containing worldwide elevation data) file:

\begin{enumerate}
\item \texttt{tar xvzf}  all tiles into one directory (taht will need more than 2 GB).
      You only need to extract the *.DEM files from the *.tar.gz archives
      downloaded from GTOPO. Use the following cmdline (in the directory with
      all archives) to extract *.DEM files only: \\
      \texttt{find . -name '*0.tar.gz' -exec tar xvzf \{\}  *.DEM ';'}

\item invoke \texttt{createworld} in the same directory (this will need another 1.8 GB)

\item enjoy the 1.8 GB \texttt{WORLD.DEM} (check \texttt{WORLD.DEM} for its size:
      should be 1.866.240.000 bytes)

\item settings for the \texttt{WORLD.DEM} can be found in the default-config file

\end{enumerate}

Now there is an improved SRTM30 model, which is based on the shuttle radar topography
mission. Basically, the SRTM30 seems to be a better GTOPO30. The SRTM30 data is also available
for free  and can be used to build the \texttt{WORLD.DEM} as described above. SRTM30 data didn't cover
the regions south of 60$^\circ$S. To obtain a \texttt{WORLD.DEM} file you should take the 6 arctic
tiles from GTOPO30, the remaining 27 from SRTM30.


\subsubsection{ETOPO2 (and merging it into the GTOPO30)}
If you have created a \texttt{WORLD.DEM} (1.8GB) datafile as described above, you can merge it with the bathymetry(sea-depth)-data from etopo20 \cite{etopo2}.
Get the \texttt{etopo20.i2.gz} file from the web. `Gunzipped' it has 116.672.402 bytes.
Put the \texttt{WORLD.DEM} and the \texttt{etopo2.i2} in the same directory and call (within that directory) \texttt{etopo2merger}, which will merge them into a \texttt{WORLD3.DEM} file.
Because the ETOPO2 resolution is lower than the resolution from GTOPO30, the additional data-points are obtained by interpolation.

\subsubsection{GLOBE}
Download the region you need (freely selectable \cite{globe}) and make sure that you
get the right data format. In the *.hdr file (which you will get too,
you can find all needed information to edit the config-file.

These are the options to be selected at GLOBE download page:\\
FreeForm ND\\
int16\\
Mac/Unix Binary\\

the data file is called   *.bin
the *.hdr file contains some information you need to edit the
ogie configfile.


\subsubsection{SRTM30 Plus (TOPO30)}
The SRTM30 Plus \cite{srtm30plus} elevation model is a merged SRTM30 and GTOPO30, including bathymetry data from several sources.
It can be downloaded as a single (1.8GB) data file from \cite{srtm30plus}.
Notice the different settings for DEM\_LAT\_MAX and DEM\_LON\_MIN! (differing from what should be used for SRTM30 and GTOPO30 world files).
See example configuration file.

\subsubsection{SRTM-1 and SRTM-3}
SRTM-3 (3 arc-seconds) data is available for free (for north and south-America and for Eurasia). SRTM-1
(1 arc-second) is available for the USA. The data is in .hgt format which is exactly, what ogie
can read. But the data is tiled into 1x1 degree pieces. This might be useful for high resolution analysis of
some terrain detail, but is just too much data for normal (glider-)flight analysis.
However, you'll find it at \cite{srtmv2}.
The Documentation folder will give you important information about data-format etc.


\subsubsection{SRTM-1 and SRTM-3 finished from seamless server}

From the usgs seamless server \cite{seamless} you can get these data.
It can be downloaded in a binary .bil format, which is accompanied by a .blw file, which contains additional information. Attention, there is a half-pixel shift.
The actual coordinates for the upp-left corner can be found in the last two lines in the .blw file.
It seems that void areas are set to 0, in contrast to the original \emph{research grade SRTM data} which uses -32768.
Additionally you will need \texttt{BIGENDIAN false}.


\subsubsection{USGS DEM (30-m and 10-m)}
\emph{This section is written by} \textsc{Vit Hradecky,} \emph{thanks} \\

Digital elevation data with 30-m and 10-m resolution for the U.S. is
now available for free at \cite{geocomm}.
The data is broken up into the standard USGS 7.5-min quads. Most of the
data is in the newer SDTS format, while some of it is in the older ASCII
DEM format. Fortunately, a utility exists to dump either into a raw
binary file, which is readable by ogie. Compile the C source from
\cite{readdem} and execute \\
\texttt{read\_dem berlin10m.DEM.SDTS.TAR berlin10m.BIN berlin10m.HDR 0} \\
This will convert data for the Berlin USGS quad from the SDTS format to
the 16-bit binary format, output to berlin10m.BIN, dump the headers into
berlin10m.HDR, and set bad data to zero elevation. The output will be in
little-endian byte order. Use the \texttt{BIGENDIAN false} option in the
configuration file to read the data correctly. The DEM latitude and
longitude limits can be found in berlin10m.HDR.
The 10-m and 30-m data is in units feet rather than meters. Use \texttt{DEM\_INPUT\_FACTOR
0.30488} in the configuration file.


\subsubsection{Configuring ogie for DEM}
\label{demconf}

For details on the configuration file see \ref{config}.
This section will only describe the settings for the digital elevation model setup.

In the configuration file you need to specify the following lines: \\
The full path to the used digital elevation data file: \\
\texttt{DEM\_FILE  /full/path/to/demfile/W020N90.DEM} \\

The number of rows and columns of data \\
\texttt{DEM\_ROWS 6000} \\
\texttt{DEM\_COLUMNS 4800} \\

The maxima and minima of your DEM-File \\
\texttt{DEM\_LAT\_MIN 40} \\
\texttt{DEM\_LAT\_MAX 90}\\
\texttt{DEM\_LON\_MIN -20}\\
\texttt{DEM\_LON\_MAX 20}\\

And the resolution (0.00833333 for GTOPO30 and GLOBE)\\
\texttt{DEM\_GRID\_LAT 0.008333333333}\\
\texttt{DEM\_GRID\_LON 0.008333333333}\\
Divide by 10 for SRTM-3.

For other config-file options see \ref{config}.
