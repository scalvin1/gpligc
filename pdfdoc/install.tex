\section{Installation}

\subsection{General Linux and Unix installation procedure}
\label{unix_install}
\label{linux_install}

This applies to all Linux and Unix operating systems.

\begin{enumerate}

\item Extract the archive: \\
\texttt{tar xvzf gpligc-version.tar.gz} \\
Change to the just created directory: \\
\texttt{cd gpligc-version}

\item Configure and build the software:\\
\texttt{./configure}\\
for options and details on configuring the build see README and the output of \texttt{./configure --help}

\item Build the software:\\
\texttt{make}

\item  \label{root}
Become root or run the next command using sudo.\\
\texttt{make install}

\item copy the example configuration file
\texttt{.ogierc} (PREFIX/share/gpligc/) to your HOME directory
and edit it according to your needs (see section~\ref{config}).

\item Make sure that Gnuplot \cite{gnuplot} is installed and in the path.
GPLIGC will also work without Gnuplot,  but you will not be able to use the plotting features.

\item Make sure that the Perl/Tk \cite{perltk} module is installed

\item Read the documentation to learn how to use gpligc \& ogie

\end{enumerate}


\subsubsection{Dependencies on ubuntu}
As I get many questions from ubuntu users because of missing dependencies, here is a list of
packages, which may need to be installed (maybe some more experienced ubuntu user can verify this list and narrow it down to what is really needed...)

To compile the package the following packages may be needed:
\texttt{libjpeg-dev}, \texttt{freeglut3-dev},
\texttt{libgl1-mesa-dev},
\texttt{libglu1-mesa-dev}. For osmesa-support add \texttt{libosmesa-dev}.

For gpsd-support add \texttt{libgps-dev}, \texttt{libqgpsmm-dev} and \texttt{gpsd}.

To run GPLIGC \texttt{perl-tk}, \texttt{libimager-perl}, \texttt{libimage-exiftool-perl}.

Gnuplot: \texttt{gnuplot (gnuplot-x11, gnuplot-qt)}.


%\subsection{Mac OS X}
\label{mac}
This paragraph is mainly written by Matthew Hoover. Also he was the one, who did the testing while gpligc/ogie was ported to Mac OS X.
Owing to his efforts, binaries for ppc- and intel-based Mac OSX machines can be provided.
Thanks a lot, Matthew!

Another section was added by Michael Schlotter, who describes how to install ogie/ogie without using fink.
For that reason Perl/Tk and Gnuplot have to been build from sources. Thanks Michael!


\subsubsection{General}
In order to get gpligc running, you need Perl with the Perl/Tk module. This usually requires to have X11 installed, since Perl/Tk doesn't work with the native Mac OSX GUI (Aqua).
As you also have to install gnuplot, I recommend the easy way shown in section~\ref{fink} using the fink platform.
If you're a more experienced developer (not using fink), you may want to compile Perl/Tk and gnuplot by yourself.
See section~\ref{schlotter}.
To compile the ogie binaries on your machine, you'll need some developer tools (as gcc, make, etc.).
So far, I didn't find a way not to use the jpeglib from fink (if you know, please report).


\subsubsection{Matthew's howto, using fink}
\label{fink}
\begin{enumerate}

\item  Install X11 from the Tools disk included with the OSX package. \\
(X11 is the unix X-Window system and is needed for the  Perl/Tk-stuff (for gpligc)).

\item  Install fink \cite{fink}. \\
Fink is a packaging system that allows Mac users to install, compile and use a wide range of free software.
We need this to install gnuplot and the Perl/Tk-module.

    \begin{enumerate}
        \item  If the user is unfamiliar with fink then also install fink
                commander which is a GUI front end for fink.
        \item  Make sure that the fink or fink commander is able to install
            "unstable" packages.
            You have to modify \texttt{/sw/etc/fink.conf} (Add main/unstable to the line containing "Trees:")
            More can be found in the fink documentation.
    \end{enumerate}

\item  Install and compile the Perl/Tk module. Therefore you have to check which Perl is installed on your system (\texttt{perl -V}).
    There are different Perl/Tk-module versions (and I am not sure if the newer ones will work with older perls...)
    According to your installed perl choose one of tk-pm560, tk-pm580 or tk-pm581. This example will continue with tk-pm581.\\

Install the tk-pm581 package using fink (\texttt{fink install tk-pm581}) or fink
commander.  There are also tk-pm581-bin and tk-pm581-man.  These will
be installed and compiled and archived when the first is selected.  If
gpligc will not run try unpacking these also.

\item  Install gnuplot using fink (\texttt{fink -b install gnuplot}) or fink commander.  I used the last
stable version with the binary files because it is faster although
there is a newer unstable version.

\item Now you can proceed with the instructions for Unix (see \ref{unix_install}) installation. Be aware to use a binary package for Mac OS X. 
The installation paths will be different from those used in Unix/Linux instructions.
For the installation prefix  you can choose by yourself (maybe /sw or /usr).

\item  To run, open X11 and then Terminal.  GPLIGC must be started from
within Terminal and X11 must also be running because it will not be
automagically started.

\item You should add \\
\texttt{test -r /sw/bin/init.sh \&\& . /sw/bin/init.sh} \\
to your \texttt{.bashrc} file to start gpligc in a xterm shell
without a terminal shell opened.

\end{enumerate}


\subsubsection{Michael Schlotter's howto}
\label{schlotter}

\begin{enumerate}
\item  Install Developer Tools from the Mac OSX Installation CD/DVD
\item  Download and install Apples X11-Server
\item  Install X11SDK and BSDSDK. This is done by double-clicking on
    \texttt{X11SDK.pkg} and \texttt{BSDSDK.pkg} in
    \textit{/Applications/Installers/Developer Tools/Packages}
\item Download, build and install gnuplot 4.0.0.
    Don't worry if \textit{make-check} produces some errors.
\item Download, build and install Perl/TK. I used \texttt{Tk-804.027.tar.gz}.
    Don't worry if \textit{make-test} produces some errors.
\item Download and install the latest gpligc with precompiled binaries for Mac
\end{enumerate}


\subsection{Windows XP/Vista/Win7/Win8}
\label{windows_install}


You'll need Perl. If unsure which Perl distribution to use, read section~\ref{perl}. But you need Perl before you proceed!

\begin{enumerate}

\item Unzip the GPLIGC-version-win32.zip archive to a temporary location
(maybe you have done that already).
Open this location in the Explorer and double-click the install-script: \\
\texttt{install\_windows.pl} \\
(if *.pl scripts are not associated with the perl-interpreter already,
you can try "open with", select browse and find bin/perl.exe in the
perl-install directory). \\
\textbf{Attention!} Don't run the install script from within the zip file. That method will not work!
Unpack the zip archive in any case and run the script from the unpacked directory.

\item The Installation script will ask you for a location to install.
Let the script do the following work for you:
a) copy all files to the install-location
b) set some environment variables by adding them to the registry.

% either by setting them in
%\texttt{autoexec.bat} (Win95/98/ME) or

\item If the installation script fails, and tells you to set the environment Variables
by yourself:\\
Make sure that an environment variable GPLIGCHOME is set,
which contains the full absolute path to the GPLIGC-directory:\\
For example: \\
\texttt{c:$\backslash$some$\backslash$path$\backslash$GPLIGC} \\
And add the gpligc-directory to your PATH \\
How to set an environment variable: \\
a) Windows NT, 2000, XP, Vista: \\
Start - Settings - Control Panel - System (Advanced) - Environment... \\

%b) Windows 95, 98, ME: \\
%You have to add a line to the "c:$\backslash$autoexec.bat": \\
%set GPLIGCHOME$=$C:$\backslash$some$\backslash$path$\backslash$GPLIGC

\item You can remove the temporary directory, where the zipfile was extracted.

% GNUPLOT 4.2.6. is included in the package from 1.10pre7.
% seems to be the last available version with static wgnuplot.exe
% from 4.4 wgnuplot needs a bunch of dlls.

%\item If you do not have Gnuplot installed already, do so.
%Get it from the gnuplot page \cite{gnuplot} (get the win32 zip, e.g. \texttt{gp425win32.zip}).
%Installation is simple: just put the file \texttt{bin/wgnuplot.exe} in a location which is in the path
%(or the gpligc install location) \\
%\textbf{ATTENTION!} The executable file of the older gnuplot \emph{3.7.x} is named \texttt{wgnupl32.exe}. If you use
%the older gnuplot, you have to change the setting for the configuration key \texttt{gnuplot\_win\_exec} to \texttt{wgnupl32.exe}
%in \texttt{gpligc.ini} (for details see \ref{gpligcrc}).
%However, you should consider updating to gnuplot 4.x, since there are new nice interactive features like zooming, rotating of 3d-plots etc.
%The later gnuplot version (4.6) are shipped with an installer, which is easy use.


\item Edit the configuration file \texttt{ogie.ini} if you like to use a digital
elevation model, digitised maps, waypoints and/or airspace files.

For details read sections \ref{dem}, \ref{maps}, \ref{wp} and \ref{airspace}.

\item Create a shortcut to \texttt{GPLIGC.pl} on your desktop if you like

\end{enumerate}


%\paragraph{Notes for Windows 95, Windows ME}
%Support for these platforms is discontinued.

%GPLIGC has not been tested with Windows 95 and ME. On Win95 you probably will need the OpenGL libraries from Microsoft, because the early win95 versions do not have OpenGL support. Windows ME should work (maybe someone
%can verify that and write an email to me?)
%\paragraph{Notes for Windows 95, 98 and ME}
%These platforms will not be supported in the future.
%The cygwin environment (which is used to build the package on windows systems) will soon be updated from version 1.5 to 1.7.
%With that update support for the older windows systems will be discontinued.
%If you want to use openGLIGCexplorer on old windows systems, you should maintain a working cygwin 1.5 installation to build openGLIGCexplorer.


\subsubsection{Perl on Windows}
\label{perl}
There are two (probably even more) important Perl distribution for windows systems:
(1) Strawberry Perl \cite{strawberryperl}, which is a open-source distribution, with an easy-to-use installer.
I personally use this and recommend its use for gpligc.
(2) ActiveState ActivePerl~\cite{activeperl}, which is a closed-source distribution (but free for personal use).
On ActiveState Perl you can use the ppm package manager to install the needed modules (Tk, Imager, Image::ExifTool).

\paragraph{Strawberry Perl}
After downloading the msi installer package of Strawberry Perl (see~\ref{requirements} for specific version),
the installation is straight forward.
Then, open a cmd.exe command-window (or perl commandline from the strawberry program folder) and enter the following
command to install perl Tk:\\
\texttt{ppm install Tk}\\
You can also use CPAN to install Tk, like the other modules below (however, this will take a couple of minutes more, as it will compile the Tk module from scratch).
To install the additional modules use the CPAN client (strawberry perl / tools) and enter the following commands
at the CPAN promt:\\
\texttt{install Image::ExifTool}\\
\texttt{install Imager}  $\leftarrow$ in recent strawberry perl this is included already.\\
Done!
Anti virus software may need to be disabled during CPAN installs (caused errors on my system).


%\paragraph{ActiveState ActivePerl}
%You can't use the new cool map-feature of gpligc if you decide to use this one! (If you're a more experienced Perl/Windows user, you may try to fix the lack of png support in the Imager module, or maybe ActiveState will ship a proper-built Imager module one day... --- let me know).
%Download and install the ActiveState ActivePerl from \cite{activeperl}.
%If your Active State Perl version is later or equal 5.10, you have to open the Perl Package Manager and install the Tk (804.029) module.
%Tk is not shown in the default list, you have chose `all packages' from the view menu.




\subsection{Additional Perl modules}
For best experience with gpligc you should install the following Perl modules:\\

\begin{itemize}
 \item \texttt{Image::ExifTool} \quad needed for photo-locator and geo-tagging. See \cite{exiftool}.
 \item \texttt{Imager} \quad needed for maps. See \cite{imager}.
\end{itemize}

there are (at least) two ways of installing Perl modules

\subsubsection{manually} You should go to the CPAN \cite{cpan} and search for the modules, download and install them.
 After downloading the archive(s), it takes the usual three commands: \\
\texttt{perl Makefile.PL} \\
\texttt{make} \\
\texttt{make install}   (as root)\\


\subsubsection{using the CPAN.pm module} If the cpan module isn't configured yet, this can be done interactively or even automated during this process. \\
\texttt{perl -MCPAN -e shell} \\
then enter \\
\texttt{install Image::ExifTool}\\
at the cpan prompt.
