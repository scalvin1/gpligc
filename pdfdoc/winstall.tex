\subsection{Windows XP/Vista/7/8}
\label{windows_install}
You'll need Perl. If unsure which Perl distribution to use, read section~\ref{perl}. But you need Perl before you proceed.

\begin{enumerate}
\item Unzip the GPLIGC-version-win32.zip archive to a temporary location
(maybe you have done that already).
Open this location in the Explorer and double-click the install-script: \\
\texttt{install\_windows.pl} \\
(if *.pl scripts are not associated with the perl-interpreter already,
you can try `open with', select browse and find bin/perl.exe in the
perl-install directory).\\
\textbf{Attention:} Don't run the install script from within the zip file. That method will not work!
Unpack the zip archive in any case and run the script from the unpacked directory.

\item The Installation script will ask you for a location to install.
Let the script do the following work for you:
a) copy all files to the install-location
b) set some environment variables by adding them to the registry.

\item If the installation script fails, and tells you to set the environment Variables
by yourself:\\
Make sure that an environment variable GPLIGCHOME is set,
which contains the full absolute path to the GPLIGC-directory:\\
For example: \\
\texttt{c:$\backslash$some$\backslash$path$\backslash$GPLIGC} \\
And add the gpligc-directory to your PATH \\
How to set an environment variable: \\
a) Windows NT, 2000, XP, Vista: \\
\menu{Start>Settings>Control Panel>System>(Advanced)>Environment}

\item You can remove the temporary directory, where the zipfile was extracted.

\item Edit the configuration file \texttt{ogie.ini} if you like to use a digital
elevation model, digitised maps, waypoints and/or airspace files.

For details read sections \ref{dem}, \ref{maps}, \ref{wp} and \ref{airspace}.

\item Create a shortcut to \texttt{GPLIGC.pl} on your desktop if you like

\end{enumerate}


\subsubsection{Gnuplot}
The gpligc package contains a statically linked binary of gnuplot 4.2.6. 
However, if you want to use a more recent version, just delete the \texttt{wgnuplot.exe} file from the GPLIGC folder.
Another method is to set the full path to your wgnuplot.exe with \texttt{gnuplot\_win\_exec} in \texttt{gpligc.ini} (see section~\ref{gpligcrc}).

\subsubsection{Perl on Windows}
\label{perl}
There are two (probably even more) important Perl distribution for windows systems:
(1) Strawberry Perl \cite{strawberryperl}, which is a open-source distribution, with an easy-to-use installer.
I personally use this and recommend its use for gpligc.
(2) ActiveState ActivePerl~\cite{activeperl}, which is a closed-source distribution (but free for personal use).
On ActiveState Perl you can use the ppm package manager to install the needed modules (Tk, Imager, Image::ExifTool).
However, the Imager module shipped with ActivePerl is currently \textbf{broken and does not work}.


\paragraph{Strawberry Perl}
After downloading the msi installer package of 32bit Strawberry Perl (see~\ref{requirements} for specific version),
the installation is straight forward.
% (unfortunately, there are no package repositories for 64bit stuff, consequently the 64bit installation does \emph{not} have ppm. You'll have to use the 32bit version. On 64bit perl, the only option to install additional modules is CPAN).
%Then, open a cmd.exe command-window (or perl commandline from the strawberry program folder) and enter the following
%command to install perl Tk:\\
%\texttt{ppm install Tk}\\
%You can also use CPAN to install Tk, like the other modules below (however, this will take a couple of minutes more, as it will compile the Tk module from scratch).
To install the additional modules use the CPAN client (strawberry perl / tools) and enter the following commands
at the CPAN promt:\\
\texttt{install Tk}\\
\texttt{install Image::ExifTool}\\
\texttt{install Imager}  $\leftarrow$ in recent strawberry perl this is included already.\\
Anti virus software may need to be disabled during CPAN installs (caused errors on my system).
