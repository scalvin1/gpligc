\subsection{Airspace}
\label{airspace}

If you want airspace information to be displayed, you should get an OpenAir\texttrademark\ airspace file (that's the same format as used by Winpilot) for your region and set up your \texttt{.ogierc} file.
One keyword declares  the filename of the airspace-file, another one sets the default, whether airspaces should be displayed or not. \\
\texttt{OPEN\_AIR\_FILE /path/to/OpenAir/file} \\
\texttt{AIRSPACE true} \\
An alternative way are the following command-line options \\
\texttt{--airspace-file=/path/to/OpenAir/file} and \texttt{--airspace} or \texttt{--no-airspace} to turn them on or off.

At runtime, airspaces can be switched on or off via the menu or by F9. Shift-F9 toggles the wire frame and transparent mode.

\subsubsection{How and where to get OpenAir files}
On the gpligc web-site you may find an airspace folder in the download area. Some OpenAir formatted files can be found there.
Another option is the page of J. Leibacher \cite{leibacher}.
