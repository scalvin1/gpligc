%$Id: tools.tex 3 2014-07-31 09:59:20Z kruegerh $



\section{Tools}

This is a small collection of Perl scripts, which can be useful for GPLIGC and OGIE users.



%\subsection{maploader.pl}
%
%Antonio Despite from Italy has written some small program to download digital raster maps from go.vicinity.com (there are other servers,
%which provide the same maps) and convert them to be used with openGLIGCexplorer. It also has given out the configuration to be
%used in .openGLIGCexplorerrc. I recoded this in Perl and added functionality to get maps for a given area.
%
%\texttt{./maploader.pl latmin latmax lonmin lonmax scale size map\_set\_name}
%
%latmin and latmax define the range in latitude. lonmin and lonmax the longitude range. The resolution can be chosen using the map scale.
%For streetlevel city navigation a scale of 5000 to 10000 should be used. 50000 - 100000 is useful for local flights, huge cross country flights
%will need 25000 to 500000. (If your graphics adapter has gigabytes of memory, you can even use thousands of 1:5000 maps for inter-continental flight,
% but most remote areas (everything aside from Europe and North America) are covered by low resolution maps. This means you will get out all
% information in big scale maps (1:100000, 1:250000, 1:500000) already).
%The size will be the pixel size of your maps, which should be a power of 2 (512, 1024, 2048 are useful values). But this has to be smaller as
%GL\_MAX\_TEXTURE\_SIZE as reported by \texttt{openGLIGCexplorer -q}.
%The map\_set\_name is used as identifier for openGLIGCexplorer.
%The output of the script will explain how to place the map files and how to modify your configuration file.

\subsection{loopviewer.pl}
The \texttt{loopviewer.pl} is a small script which can be used as a template to create an automatic show of a list of igc files.
It is intended to be used at gliding competitions to have a nice presentation of all flights. All pilots can enjoy every flight of the day, while having a nice cold beer in the briefing hangar (video-beamer!).
Therefore a list has to be created containing one line for every flight. Each line should contain the quoted path to the igc-file and a quoted information string: \\
\texttt{"c:$\backslash$path$\backslash$to$\backslash$file.igc"   "1. - Name - ASW20 - II - 610.34km - 102.3km/h - 1000pts"} \\
\texttt{"c:$\backslash$path$\backslash$to$\backslash$file2.igc"  "2. - Name - Ventus2 - I2 - 610.34km - 101.7km/h - 980pts"} \\
\texttt{...} \\
The configuration file should be set up to have some nice maps from the contest area, airspaces and whatever is needed. A good initial viewpoint position should be determined in some interactive run, subsequently the corresponding options can be changed in the \texttt{loopviewer.pl} script, which can be started with this simple call: \\
\texttt{loopviewer.pl  list} \\
where \texttt{list} is the file containing the above mentioned list.




\subsection{Garmin related tools}

Since I bought a Garmin Geko301, a few tools have been developed to use the Garmin's track logs etc.

I recommend to use \emph{gpspoint}  by Thomas Schank to do the data transfers to and from your Garmin device.
To handle the output from \emph{gpspoint} the following tools can be used.

\subsubsection{gpsp2igc.pl and gpsp2igcfile.pl}
\label{gpsp2igc}
This tool was developed by my brother Max, and can convert the track log output from \emph{gpspoint} (\texttt{gpspoint -dt >tracklog.gpsp})
to something like an igc-file to be opened by GPLIGC \& OGIE.

I use gpspoint 2.030521:

\texttt{gpspoint -p /dev/ttyS0 -dt | gpsp2igc.pl >out.igc}

gpsp2igcfile.pl creates IGC-file(s) with a filename corresponding to the date(s) of the recording.

%\subsubsection{wpz2gpsp.pl}
%This is some experimental script to convert zander waypoint-list to a format which can be uploaded to a Garmin device by \emph{gpspoint}.
%\texttt{wpz2gpsp.pl <in.wpz >out.gpsp}









%;;; Local IspellDict: "british"
%%% Local Variables:
%%% mode: latex
%%% TeX-master: "GPLIGC_manual.tex"
%%% End:
